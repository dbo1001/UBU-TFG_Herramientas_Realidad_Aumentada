\capitulo{4}{Técnicas y herramientas}


\section{Técnicas de desarrollo}
\subsection{CoSpaces}
Para el desarrollo en CoSpace empece creando un nuevo proyecto. Dado que el ejemplo deberá funcionaran con el MergeCube, en las primeras opciones que nos da al crear el proyecto se podrá incluir.
Una vez creado el proyecto con el plug de Mergecube, tendremos un espacio de trabajo vació con el cubo de Mergecube en el centro. Esto se trata de una manera sencilla de poder crear un espacio de trabajo, por lo que no sería necesario de un alto nivel de conocimientos.

Para añadir los elementos deseados en el entorno, cuenta con un sencillo panel desde el que seleccionar modelos que ofrece CoSpace, o la posibilidad de importar modelos 3D de una fuente externa, mediante un buscador en Google Poly o mediante una subida de archivos. CoSpaces es compatible con las extensiones obj, mtl, fbx, zip de modelos, y jpg, png, gif, svg, bmp de imágenes.

En cuanto a la manipulación de los modelos, se trata de una manipulación muy sencilla y rápida de aprender, da la posibilidad de mover, rotar y escalar el modelo . El problema de esta manipulación es que es muy simple y no tan precisa, por ejemplo no permite mover objetos en diagonal. Para casos en los que se requiera más precisión habría que usar lo movimientos introduciendo las coordenadas exactas.

Otro de los aspectos de la edición de los modelos es el cambio de colores y textura. No a todos los modelos se les puede cambiar la textura a tu gusto, el color en cambio si. Con los modelos importados hay más problemas a la hora de personalizar textura o color, pues si el modelo tiene diferentes capas, no las diferencia y cambiará todas del mismo color. 

En algunos modelos predefinidos es posible escoger y activar animaciones, o diferentes estados para ese modelo. Por ejemplo en un modelo de un coche se puede escoger si tiene algunas puertas abiertas o no. La parte negativa es que no puedes definir tus propios estados o animaciones, por lo que unicamente es posible escoger entre las opciones que esos modelos predefinidos tengan. En todos los modelos es posible añadir un bocadillo para añadir un dialogo. También es posible añadir en todos los modelos ciertas físicas, pudiendo ajustando la masa del objeto, seleccionar si será un objeto estático, y alguna opción más, tal vez no es tan completo como otros software pero cumple.

También cuenta con programación, con ella es posible establecer diferentes eventos, cambios de escala, posición o establecer algunas normas para un juego entre muchas más posibilidades. CoSpaces ofrece dos tipos de programación, por bloques y programación en TypeScript. La programación por bloques es una buena forma de iniciación, en especial para los más jóvenes, pues es bastante rápido aprender como funciona. La pega de la programación por bloques es que está limitado a los bloques que nos ofrece, es posible crear tus propios bloques pero no tiene suficiente "diseño" para ajustarse a tantas opciones como una programación por script. 
La otra posibilidad para programar que ofrece CoSpaces, se trata de elaborar script de en TypeScript. Si se tiene experiencia en programación en otros lenguajes, no resulta difícil acostumbrarse. La parte negativa que tiene esta opción es que la API de CoSpaces, no es tan completa como cabria esperar, sumado a que su "guía de usuario" podría estar mejor.





CoSpaces también da la posibilidad de crear espacios de trabajo para alumnos, en estos espacios se puede asignar tareas a los alumnos.
\subsection{Ejemplo Unity}

Para el desarrollo se ha encogido Unity como motor gráfico. Esto se debe a que es un motor gratuito y que goza de grandes características y utilidades. Dadas sus grandes características es uno de los mas utilizados, y la gran mayoría de herramientas de realidad aumentada tienen soporte para Unity.

El proceso para crear un proyecto es, desde Unity HUb escoger que tipo de proyecto crear, uno 2D, 3D, etc. En nuestro caso nos interesa un proyecto en 3D. Escogemos la ruta en la que se desea guardar el proyecto y adelante.
Una vez se ha generado el entorno deberemos hacer las configuraciones iniciales, que son cambiar el proyecto de un proyecto pensado para Windows, a Android que el entorno para el que crearemos la aplicación inicialmente. Para ello vamos a la pestaña File, Build Settings, desde aquí encontraremos la opción para que tipo de plataforma queremos que sea nuestro proyecto, en nuestro caso nos interesa Android.
El segundo paso será añadir el paquete de Vuforia para usar sus extensiones, hay dos opciones: desde la propia Unity si la versión en la que se trabaja lo tiene incluido,desde la pestaña Edit, Project Settings, en Player dentro de la opciones al final se encuentra el apartado XR Settings y desde aquí se puede activar el check de Vuforia que procederá a instalarse, o bien la segunda opción sería importando el SDK de Vuforia que se puede conseguir en su pagina de developer\footnote{\url{https://developer.vuforia.com/downloads/sdk}}. En la versión de Unity 2019.2 contiene la versión  de Vuforia 8.5.
Debido a que importe un ejemplo que ofrece la compañía, actualice la versión de Vuforia hasta la 8.6.10. Antes de intentar realizar una actualización de Vuforia, hay que tener en cuenta de que sea compatible con la versión de Unity, ya que puede darse el caso de que para actualizar Vuforia sea necesario también actualizar la versión de Unity, y esto si el proyecto ya está avanzado podría conllevar problemas de compatibilidad.
En mi caso he decido seguir en la versión 2019.2... de Unity y 8.6.1 de Vuforia.

El proceso para instalar el paquete de Vuforia y iniciar un proyecto no es muy complicado, por lo que prácticamente cualquier usuario siguiendo los pasos.

El segundo paso en la aplicación sería empezar a construirla. Para empezar en el proyecto los dos elementos mas importantes serán incluir la cámara de realidad aumentada de Vuforia, y el multitarget que podamos diseñar para reconocer el MergeCube. El proyecto creado por Unity lleva por defecto una cámara normal, como para nuestra aplicación no la utilizaremos la podemos borrar.
Para añadir la cámara hacemos click derecho, Vuforia Engine, y escogemos AR Camera.

Un problema que tiene el realizar una aplicación para móviles, es que cada vez que se quiera hacer una prueba en el dispositivo es necesario compilar al apk .... Este proceso para hacer pruebas puntales no tiene un gran impacto en cuanto esfuerzo y tiempo invertido, pero si es necesario hacer muchas pruebas, por que por ejemplo hay un bug especifico que unicamente ocurre en el dispositivo móvil, el impacto en tiempo invertido en compilar será considerable.


-

-

-

Para la implementación de las funcionalidades de Device Tracker y crear target personalizados, me basare en los que ofrece Vuforia en su ejemplo\footnote{\url{https://assetstore.unity.com/packages/templates/packs/vuforia-core-samples-99026}} utilizando de base sus mismos scripts, adaptando les a las necesidades del ejemplo desarrollado.

Para la captura de un target personalizado, hay que crear en la interfaz una señal que indique que el objetivo que se desea transformar en un target tiene una calidad suficiente. Esto puede realizarse mediante una representación de tres colores, como un semáforo. Después hay que añadir en el proyecto un ImagenTarget y configurarlo en el Inspector y definirlo como tipo de User Defined. Es muy importante asegurase de que el ImagenTarget está situado en el campo de visión de la ARcamera, o de lo contrario será imposible reconocerle. Seguido hay que añadir CameraImageBuilder que es el encargado de construir el target temporal. Ahora el paso ha seguir es de asginar a los componentes creados sus respectivos scripts, durante este proceso en mi primer intento Unity tuvo algún fallo y corrompió el proyecto, haciendo imposible asociar ningún script y ademas los que estaban asociados anteriormente también fallaban. Para solucionar el bug probé dos posibles soluciones: borrar los metadatos y reimportar todo el proyecto, por desgracia ninguno funciono por lo que tuve que volver a una copia de seguridad.
Para implementar el Device tracker hay que crear un botón para poder activarlo y desactivarlo, y añadir el correspondiente script de deviceTracker.




\section{Herramientas de documentación}
\subsection{LaTex}
Se trata de un sistema de composición de textos, con el objetivo de la creación de documentos de alta calidad tipográfica.
\section{Herramientas de gestión}
\subsection{GitHub}
Se trata de una plataforma donde se pueden guardar con un sistema de control de versiones los proyectos....

 Link del repositorio: \url{https://github.com/smi0010/TFG_Herramientas_Realidad_Aumentada}

\section{Herramientas de desarrollo}
\subsection{Unity}
Unity\footnote{\url{https://unity.com/es}}, es un motor gráfico de desarrollo de videojuegos creado por Unity technologies, disponible para Windows,Linux y Mac OS. En el, se pueden desarrollar aplicaciones para distintas plataformas como, Windows, MacOS,Linux, Andorid, iOS, videoconsolas, WebGL, tvOS y Facebook.
\subsection{VisualStudio}
ViusalStudio \footnote{\url{https://visualstudio.microsoft.com/es/}}, se trata de un entorno de desarrollo para Windows, Linux y macOS. Es compatible con múltiples lenguajes de programación. C\# es uno de los principales lenguajes con los que es compatible, y dado que los scripts de Unity son en dicho lenguaje, se ha escogido visual studio para su desarrollo.

\subsection{Vuforia}
Como se ha explicado en el apartado \ref{sub:Def_Vuforia}, Vuforia\footnote{\url{https://developer.vuforia.com/}} se trata de un SDK para aplicaciones de AR, para las plataformas de Android, iOS, Windows, Unity y HoloLens. Proporciona las herramientas necesarias para desarrollar realidad aumentada en el entorno Unity.
\subsection{CoSpace}
Como se exponía en el apartado \ref{sub:Def_Vuforia}, CoSpaces\footnote{\url{https://cospaces.io/edu/}} se trata de  es una plataforma de apoyo educativo. Se puede acceder a ella tanto por web, como desde la aplicación de móvil. Permite crear entornos virtuales y de realidad aumentada para el MergeCube, desde su entorno de desarrollo web.

\section{title}
