\capitulo{4}{Técnicas y herramientas}


\section{Técnicas de desarrollo}
\subsection{CoSpaces}
Para el desarrollo en CoSpace empece creando un nuevo proyecto. Dado que el ejemplo deberá funcionaran con el MergeCube, en las primeras opciones que nos da al crear el proyecto se podrá incluir.
Una vez creado el proyecto con el plug de Mergecube, tendremos un espacio de trabajo vació con el cubo de Mergecube en el centro. Esto se trata de una manera sencilla de poder crear un espacio de trabajo, por lo que no sería necesario de un alto nivel de conocimientos.

Para añadir los elementos deseados en el entorno, cuenta con un sencillo panel desde el que seleccionar modelos que ofrece CoSpace, o la posibilidad de importar modelos 3D de una fuente externa, mediante un buscador en Google Poly o mediante una subida de archivos. CoSpaces es compatible con las extensiones obj, mtl, fbx, zip de modelos, y jpg, png, gif, svg, bmp de imágenes.

En cuanto a la manipulación de los modelos, se trata de una manipulación muy sencilla y rápida de aprender, da la posibilidad de mover, rotar y escalar el modelo . El problema de esta manipulación es que es muy simple y no tan precisa, por ejemplo no permite mover objetos en diagonal. Para casos en los que se requiera más precisión habría que usar lo movimientos introduciendo las coordenadas exactas.

Otro de los aspectos de la edición de los modelos es el cambio de colores y textura. No a todos los modelos se les puede cambiar la textura a tu gusto, el color en cambio si. Con los modelos importados hay más problemas a la hora de personalizar textura o color, pues si el modelo tiene diferentes capas, no las diferencia y cambiará todas del mismo color. 

En algunos modelos predefinidos es posible escoger y activar animaciones, o diferentes estados para ese modelo. Por ejemplo en un modelo de un coche se puede escoger si tiene algunas puertas abiertas o no. La parte negativa es que no puedes definir tus propios estados o animaciones, por lo que unicamente es posible escoger entre las opciones que esos modelos predefinidos tengan. En todos los modelos es posible añadir un bocadillo para añadir un dialogo. También es posible añadir en todos los modelos ciertas físicas, pudiendo ajustando la masa del objeto, seleccionar si será un objeto estático, y alguna opción más, tal vez no es tan completo como otros software pero cumple.

También cuenta con programación, con ella es posible establecer diferentes eventos, cambios de escala, posición o establecer algunas normas para un juego entre muchas más posibilidades. CoSpaces ofrece dos tipos de programación, por bloques y programación en TypeScript. La programación por bloques es una buena forma de iniciación, en especial para los más jóvenes, pues es bastante rápido aprender como funciona. La pega de la programación por bloques es que está limitado a los bloques que nos ofrece, es posible crear tus propios bloques pero no tiene suficiente "diseño" para ajustarse a tantas opciones como una programación por script. 
La otra posibilidad para programar que ofrece CoSpaces, se trata de elaborar script de en TypeScript. Si se tiene experiencia en programación en otros lenguajes, no resulta difícil acostumbrarse. La parte negativa que tiene esta opción es que la API de CoSpaces, no es tan completa como cabria esperar, sumado a que su "guía de usuario" podría estar mejor.





CoSpaces también da la posibilidad de crear espacios de trabajo para alumnos, en estos espacios se puede asignar tareas a los alumnos.
\subsection{Ejemplo Unity}

\section{Herramientas de documentación}
\subsection{LaTex}
Se trata de un sistema de composición de textos, con el objetivo de la creación de documentos de alta calidad tipográfica.
\section{Herramientas de gestión}
\subsection{GitHub}
Se trata de una plataforma donde se pueden guardar con un sistema de control de versiones los proyectos....

 Link del repositorio: \url{https://github.com/smi0010/TFG_Herramientas_Realidad_Aumentada}

\section{Herramientas de desarrollo}
\subsection{Unity}
Unity\footnote{\url{https://unity.com/es}}, es un motor gráfico de desarrollo de videojuegos creado por Unity technologies, disponible para Windows,Linux y Mac OS. En el, se pueden desarrollar aplicaciones para distintas plataformas como, Windows, MacOS,Linux, Andorid, iOS, videoconsolas, WebGL, tvOS y Facebook.
\subsection{VisualStudio}
ViusalStudio \footnote{\url{https://visualstudio.microsoft.com/es/}}, se trata de un entorno de desarrollo para Windows, Linux y macOS. Es compatible con múltiples lenguajes de programación. C\# es uno de los principales lenguajes con los que es compatible, y dado que los scripts de Unity son en dicho lenguaje, se ha escogido visual studio para su desarrollo.

\subsection{Vuforia}
Como se ha explicado en el apartado \ref{sub:Def_Vuforia}, Vuforia\footnote{\url{https://developer.vuforia.com/}} se trata de un SDK para aplicaciones de AR, para las plataformas de Android, iOS, Windows, Unity y HoloLens. Proporciona las herramientas necesarias para desarrollar realidad aumentada en el entorno Unity.
\subsection{CoSpace}
Como se exponía en el apartado \ref{sub:Def_Vuforia}, CoSpaces\footnote{\url{https://cospaces.io/edu/}} se trata de  es una plataforma de apoyo educativo. Se puede acceder a ella tanto por web, como desde la aplicación de móvil. Permite crear entornos virtuales y de realidad aumentada para el MergeCube, desde su entorno de desarrollo web.

\section{title}
