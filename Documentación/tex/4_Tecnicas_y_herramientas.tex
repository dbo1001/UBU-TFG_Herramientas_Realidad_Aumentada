\capitulo{4}{Técnicas y herramientas}

\section{Herramientas}
\subsection{Herramientas AR}

Se han comparado diferentes herramientas de desarrollo para realidad aumentada. 
Algunas de ellas se tratan de herramientas ya desarrolladas y usadas con fines educativos o comerciales. Otras de las herramientas que se explican son los frameworks/API que muchas de las herramientas anteriores usan.


\subsubsection{ArCore}

ArCore \footnote{\url{https://developers.google.com/ar/discover}} es la plataforma de Google de desarrollo para realidad aumentada. Con el uso de diferentes API permite que los dispositivos detecten su entorno, lo comprendan e interactúen con la información.

Para conseguir integrar el contenido virtual en el mundo real, ArCore utiliza tres técnicas fundamentales:
\begin{itemize}
	\item Motion Tracking (seguimiento del movimiento): permite establecer la posición del móvil en relación con el mundo.
	\item La compresión ambiental: esta permite detectar la posición y tamaño de las superficies del entorno.
	\item Estimación de la luz: que permite calcular las condiciones de luz del ambiente.
\end{itemize}

Su funcionamiento se puede resumir en dos pasos principalmente: rastrear la posición del dispositivo a medida que se mueve y construir su propia comprensión del mundo real.

ArCore utiliza el Motion Tracking para identificar los puntos clave, y rastrea cómo esos puntos se mueven. Combinado el movimiento de dichos puntos con  las lecturas de los sensores de inercia del dispositivo, ArCore calcula la posición y orientación del dispositivo mientras se mueve~\cite{google}.

Lamentablemente no es compatible con cualquier dispositivo. En el siguiente enlace se encuentra la lista de los dispositivos compatibles actualmente \url{https://developers.google.com/ar/discover/supported-devices}.

\subsubsection{ArKit}

ARKit \footnote{\url{https://developer.apple.com/augmented-reality/}} es la herramienta de realidad aumentada de Apple para sistemas iOS. ARKit consigue mostrar contenido virtual de forma natural en el mudo real, incluso pudiendo situarlo detrás o delante del usuario con People Occlusion, pudiendo reconocer hasta 3 rostros al mismo tiempo~\cite{apple_inc}.

Funciones destacadas que posee ArKit actualmente.

\begin{itemize}
	\item Oclusión de personas: es capaz de diferenciar a una persona del fondo del escenario, consiguiendo que el contenido virtual  pueda pasar por delante o por detrás de la persona.
	\item Captura de movimiento: Tiene la capacidad de capturar los movimientos de una persona. Distinguiendo diferentes posiciones y movimientos al instante, de forma que es capaz de usarlo como referencias para experiencias de AR.
	\item Cámara frontal y trasera simultánea: Tiene la capacidad de usar ambas cámaras al mismo tiempo, pudiendo así por ejemplo interactuar con el entorno capturado por la cámara trasera, usando únicamente el rostro.
	\item Seguimiento de rostros múltiples: Es capaz de reconocer hasta 3 rostros al mismo tiempo usando la cámara TrueDepth.
	\item Sesiones colaborativas: Tiene la capacidad de crear un mapa mundial, entre múltiples usuarios conectados. Siendo así capaz de crear experiencias de realidad aumentada mas rápido, que pueden servir por ejemplo para juegos multijugador.
\end{itemize}



Una limitación es que únicamente esta disponible para sistemas iOS 11.0 o superior.


\subsubsection{CoSpaces}\label{sub:Def_cospace} 

CoSpaces\footnote{\url{https://cospaces.io/edu/}} es una plataforma de apoyo educativo. Se puede acceder a ella tanto por web, como desde la aplicación de móvil. Permite a los profesores crear salas para sus alumnos donde tendrían acceso a los diferentes ejemplos a usar para sus clases. También cuenta con una galería donde los usuarios pueden compartir sus ejemplos.

\imagen{cospaces2}{Imagen de mi plataforma de desarrollo de CoSpaces. ( Captura de pantalla)}

Para la creación de proyectos, se puede hacer desde la web mediante un entorno de desarrollo 3D, donde puedes añadir paredes, objetos y mas tipos de modelos3D, para tu escenario. Con una licencia gratuita, permite guardar dos proyectos como máximo al mismo tiempo, en la imagen~\ref{fig:cospaces2} se puede observar el entorno descrito. En los proyectos también es posible codificar eventos, acciones mediante una programación de bloques o scripts, aunque hay que destacar, que la programación por scripts y algunas opciones de la programación de bloques únicamente están disponibles con una licencia premium. Para añadir un modelo 3D, que Cospace no ofrezca por defecto, se pueden subir desde un archivo local o desde la búsqueda integrada de CoSpace en Google Poly (biblioteca pública de google de modelos 3D\footnote{\url{https://poly.google.com/}}).
CoSpace también cuenta con la opción de diseñar proyectos para el Mergecube, aunque este plugin se encuentra disponible para la versión premium.



\subsubsection{Metaverse} 

Metaverse\footnote{\url{https://studio.gometa.io/discover/me}} es otra aplicación enfocada bastante al entorno educativo. En este caso el entorno de desarrollo de los ejemplos es más sencillo. En la propia web, tendremos una estructura similar a un modelo de diagrama de flujo. En cada paso se puede añadir un objeto 3D o una imagen que es la que estará flotando cuando estemos usando la realidad aumentada, posteriormente se pueden añadir botones y menús, para que te lleven a otra pantalla que tenga otro objeto asignado. Así, por ejemplo, se pueden hacer programas en los que se creen pequeños juegos de preguntas y, dependiendo de las respuestas, te llevan a diferentes pantallas.
Además da la posibilidad de importar tus propios modelos e imágenes.
También incluye la posibilidad de reconocer expresiones faciales para poder usarlo como disparadores.
Otra característica es que puede usar el ArCore y ArKit para permitir que los objetos puedan asentarse en una posición y poder girar alrededor suyo.

\subsubsection{Zapworks} ZapWorks\footnote{\url{https://zap.works/}} se trata de una herramienta de desarrollo para AR. En general esta compuesto por dos herramientas de creación, ZapWorks Designer y ZapWorks Studio.

\begin{itemize}
	\item ZapWorks Studio: se trata de la aplicación de desarrollo para escritorio. La aplicación permite desarrollar los proyectos AR, con modelos 3D, animaciones, interacciones, etc. Soporta los formatos  GLTF, FBX, OBJ, POD de modelos 3D. En la imagen~\ref{fig:zapworkStudio} se puede observar el entorno.
	\item ZapWorks Designer: Se trata de una aplicación web, desde la que se puede de una forma sencilla, asociar a un marcador, imágenes, textos, vídeos, links. Pero no es posible trabajar con modelos 3D.
\end{itemize}	

\imagen{zapworkStudio}{Imagen perteneciente a la aplicación de ZapWorks Studio~\cite{ZapStudio}.}

La herramienta posee los siguientes tipos de seguimiento de realidad aumentada.

\begin{itemize}
	\item Word Tracking: Capacidad de reconocer el entorno para poder colocar objetos de una manera mas realista sobre el terreno, sin la ayuda de marcadores.
	\item Face Tracking: Capacidad de reconocer y seguir rostros, para por ejemplo colocar, en una cara el contenido AR y que aunque se muevan el contenido se mueva con la cara.
	\item Image Tracking: Que una imagen pueda ser el marcador, y pueda seguirla en movimiento.
	\item Zapbox Tracking: Se trata de unos marcadores específicamente creados para que se puedan manipular~\ref{fig:Memoria/zapbox}.
\end{itemize}

\imagen{Memoria/zapbox}{Imagen perteneciente a un kit de zapbox~\cite{ZapStudio}.}

Para acceder al contenido AR almacenado en el servidor de ZapWorks, utiliza los zapcodes~\ref{fig:zapcode}, se tratan una imagen codificada para que pueda descargar un proyecto AR. Una vez que la aplicación reconoce el zapcode, descarga el proyecto y este se ejecutará.

\imagenPeque{zapcode}{Ejemplo de un zapcode~\cite{ZapStudio}.}


\subsubsection{Kudan} 

Kudan\footnote{\url{https://www.xlsoft.com/en/products/kudan/index.html}} 
se trata de un SDK, con la capacidad de soportar AR tanto con marcadores y como sin ellos. En principio Kudan no tiene un limite de marcadores que pueda detectar al mismo tiempo, pero se puede incluir un limite para que la aplicación tenga un rendimiento adecuado. 
Posee soporte para APIs nativas, como ObjetiveC para iOS, Java para Android y Unity~\cite{kudan_developer_hub}.

El motor de Kudan está escrito en C++ y optimizado con programación en ensamblador, dando un mayor rendimiento y estabilidad, con el menor impacto de memoria.
Usa la tecnología SLAM (Simultaneous Localization and Mapping) para el reconocimiento del escenario.

Kudan soporta los siguientes formatos 3D: FBX, OBJ y COLLADA.

En la imagen \ref{fig:marta_2} podemos ver un ejemplo de la aplicación M.A.R.T.A\footnote{\url{https://play.google.com/store/apps/details?id=com.apophistechlabs.marta}} que utiliza Kudan para detectar una superficie en la que colocar el modelo 3D.

\imagenPeque{marta_2}{Imagen perteneciente a la aplicación M.A.R.T.A~\cite{marta_app}.}



\subsubsection{Vuforia}\label{sub:Def_Vuforia}

Vuforia\footnote{\url{https://developer.vuforia.com/}} es un kit de desarrollo ( SDK) para aplicaciones de AR, para las plataformas de Android, iOS, Windows, Unity y HoloLens. Fue creado por Qualcomm Connected Experience en 2010, y en 2015 PTC inc lo compró~\cite{simonetti2013vuforia}.

Vuforia proporciona una API en C++, Java, Objective-C++ y los lenguajes de .NET mediante Unity.

Vuforia soporta diferentes tipos marcadores~\ref{fig:vuforia01}, estos pueden ser 2D o 3D, también soporta múltiples marcadores simultáneamente, y reconocimiento del terreno sin marcadores.

\imagen{vuforia01}{Posibles marcadores de Vuforia}

Los marcadores se pueden crear desde la página web de developer vuforia, para ello deben escoger qué tipo de marcadores se quieren crear y adjuntarle las imágenes que formarán dicho marcador. La página dará una calificación de 5 estrellas según la calidad de la imagen para ser un marcador. Una vez completado, da la opción de descargarlos el un archivo configurable de Unity que incluirá los marcadores en nuestro proyecto.



Cuenta con una licencia gratuita por defecto, con la limitación de poder tener un máximo de 100 vumarks y 1000 cloud targets.

Actualmente Vuforia esta disponible para las versiones indicadas en la siguiente tabla~\cite{vuforia_supported_versions}.

REVISAR TABLA   ELIMINAR SI REALMENTE NO ES RELEVANTE
\begin{table}[]
	\begin{tabular}{|l|l|l|l|l|l|}
		\hline
		\multicolumn{2}{|l|}{Device OS}                       & \multicolumn{2}{l|}{Herramientas de Desarrollo}                                             & \multicolumn{2}{l|}{Fusion Provider}                \\ \hline
		&                          & NDK                                            & r20+                                       &                                &                    \\ \cline{3-4}
		&                          & \cellcolor[HTML]{EFEFEF}Gradle                 & \cellcolor[HTML]{EFEFEF}5.1.1+             &                                &                    \\ \cline{3-4}
		&                          & Android SDK Build Tools                        & 28.0.3                                     &                                &                    \\ \cline{3-4}
		&                          & \cellcolor[HTML]{EFEFEF}Android Studio         & \cellcolor[HTML]{EFEFEF}3.4.x              &                                &                    \\ \cline{3-4}
		\multirow{-5}{*}{Android}  & \multirow{-5}{*}{5.1.1+} & Unity Editor                                   & 2019.2.0+                                  & \multirow{-5}{*}{ArCore 1.10+} & \multirow{-5}{*}{} \\ \hline
		&                          & \cellcolor[HTML]{EFEFEF}XCode                  & \cellcolor[HTML]{EFEFEF}10.1+              & \multicolumn{2}{l|}{}                               \\ \cline{3-4}
		\multirow{-2}{*}{iOS}      & \multirow{-2}{*}{11+}    & Unity Editor                                   & 2019.2+                                    & \multicolumn{2}{l|}{\multirow{-2}{*}{ArKit}}        \\ \hline
		&                          & \cellcolor[HTML]{EFEFEF}Visual Studio          & \cellcolor[HTML]{EFEFEF}2017 v 15.9+ & \multicolumn{2}{l|}{}                               \\ \cline{3-4}
		&                          & Unity Editor                                   & 2019.2.0+                                  & \multicolumn{2}{l|}{}                               \\ \cline{3-4}
		\multirow{-3}{*}{Windows}  & \multirow{-3}{*}{10}     & \cellcolor[HTML]{EFEFEF}Unity Editor(Hololens) & \cellcolor[HTML]{EFEFEF}2018.4.11          & \multicolumn{2}{l|}{\multirow{-3}{*}{}}             \\ \hline
		&                          & Lumin SDK                                      & 0.22.0                                     & \multicolumn{2}{l|}{}                               \\ \cline{3-4}
		\multirow{-2}{*}{Lumin Os} & \multirow{-2}{*}{10}     & \cellcolor[HTML]{EFEFEF}Lumin OS               & \cellcolor[HTML]{EFEFEF}0.97+              & \multicolumn{2}{l|}{\multirow{-2}{*}{}}             \\ \hline
	\end{tabular}
\end{table}


\subsubsection{Wikitude}

Wikitude\footnote{\url{https://www.wikitude.com/}} se trata de un SDK de realidad aumentada, ofrece un amplio repertorio de características~\cite{wikitude}: 

\begin{itemize}
	\item Reconocimiento de imágenes.
	\item Reconocimiento de objetos.
	\item Reconocimiento de múltiples marcadores: Permite reconocer múltiples marcadores de forma simultánea.
	\item Instant Tracking: reconocimiento sin la necesidad de marcadores, permitiendo superponer elementos virtuales en superficies detectadas con la tecnologia Slam Instant Tracking.	
	\item Seguimiento extendido: Permite que un elemento virtual que se ha superpuesto en un punto o marcador determinado, persista aunque dicho punto se salga del campo de visión de la cámara.	
	\item Geo AR: Permite agregar contenido de realidad aumentada basándose en la ubicación, valiéndose del GPS y demás sensores para determinar la ubicación.
	\item Cloud Recognition: Ofrece la posibilidad de guardar en línea los datos de imágenes/marcadores para las aplicaciones.	
\end{itemize}

Wikitude esta disponible para Android, iOS, Windows y Smart Glasses.
También posee un amplio soporte para frameworks de desarrollo : Andoid, iOS, Windows, Unity, Cordova, Xamarin, Flutter, Titanium. 
Los sistemas deberán cumplir los requisitos indicados en el siguiente enlace: \url{https://www.wikitude.com/documentation/latest/android/supporteddevices.html#supported-devices}.



\subsubsection{OpenCV}

OpenCV\footnote{\url{https://opencv.org/}} es una librería de visión artificial y machine learning, de código abierto. Tiene una licencia BSD por lo que facilita su uso y modificación para las empresas. Cuenta con más de 2500 algoritmos optimizados. Estos pueden emplearse para detectar y reconocer rostros, identificación de objetos, clasificar acciones humanas en vídeos, seguir los movimientos de la cámara, rastrear objetos en movimiento, extraer modelos 3D de objetos, producir nubes de puntos 3D de cámaras estéreo, seguir el movimiento de los ojos, reconocer paisajes y establecer marcadores para usarlos en realidad aumentada.

Esta disponible en interfaces para C++, Python, Java y MATLAB y es compatible con Windows, Linux, Android y Mac OS.


\subsubsection{8thWall}

8thWall\footnote{\url{https://www.8thwall.com/}}
es un entorno de desarrollo de realidad aumentada, que se caracteriza principalmente por el WebAR, que da la posibilidad de ejecutar las imágenes AR a través del propio navegador web del dispositivo, sin tener que instalarse ninguna aplicación extra en el mismo~\cite{8thwall_products}.

Posee las siguientes características:

\begin{itemize}
	\item World Tracking: Con la tecnología SLAM, es capaz de reconocer superficies planas instantáneamente, ademas de estimaciones de iluminación.
	\item Image Targets: 8thWall Web puede detectar y rastrear imágenes y usarlas como marcadores. Cada aplicación puede tener un máximo de 1000 marcadores.
	\item Modular framework: El framework de la aplicación esta diseñado para integrar tecnologías de reconocimiento como el seguimiento de rostros, la oclusión de personas y otras tecnologías de machine learning.
\end{itemize}


Tiene soporte con los frameworks 3D A-Frame, three.js, babylon.js, Amazon Sumerian y PlayCanvas.

Requiere en iOS, iOS11 o superior, y el navegador Safari. En Android los navegadores soportados son; Chrome, Chromium,Firefox, Android WebViews.

\subsubsection{Resumen y comparativa de herramientas}
En la tabla \ref{tabla:comparacionHerramientas} se puede observar una comparativa entre las principales funcionalidades de las herramientas. Se indica si es MultiTarget (MT), si posee realidad sin marcadores (SM), las plataformas para las que está disponible, el tipo de licencia, libre (L) o comercial (C), el precio y tipo de requisitos.

\label{tablaComparacion}\tablaSmall{Comparativa herramientas AR}{l c c c c r c}{comparacionHerramientas}
{ \multicolumn{1}{l}{} & MT & SM & Plataforma  & Licencia & Precio & Requisitos\\}{ 
	ArCore & X&X&iOS/Android& L & 0\euro & Altos\\ 
	ArKit &X&X&iOS &L & 0\euro &-\\
	CoSpaces&X & &iOS/Android  &L y C & 75\euro/año&Medios\\
	Metaverse &X&X&iOS/Android &L y C & ?\euro&Medios \\
	Zapworks  &X&&iOS/Android &L y C & 55\euro/mes& Medios\\
	Kudan   &X&X&iOS/Android &L y C & ?\euro& Medios\\
	Vuforia  &X&X& iOS/Android&L y C & 42\euro/mes& Medios\\
	WikiTude &X&X&iOS/Android &L y C & 2490\euro & Medios\\
	OpenCV  &X& & iOS/Android&L & 0\euro& Bajos\\
	8thWall &X&X& iOS/Android&L y C & 99mes\euro& Medios\\
}

\subsection{Herramientas de documentación}
\subsubsection{\LaTeX} 
Para documentar este proyecto se ha utilizado \LaTeX. Se trata de un sistema de composición de textos, con el objetivo de la creación de documentos con una alta calidad tipográfica. Se ha usado TeXstudio y miktex para compilar y editar \LaTeX.

A diferenciara de otros editores clásicos de textos para documentación, \LaTeX tiene una curva de aprendizaje más elevada, pero al final sus ventajas a la hora de documentar, especialmente este tipo de proyectos, le convierte una gran opción.


\subsubsection{Draw.io}
Draw.io\footnote{\url{https://www.diagrams.net/}} se trata de una herramienta online, para el modelado de diagramas UML. Una de sus grandes ventajas es la posibilidad de guardar y sincronizar los diagramas en servicios como Google Drive, OneDrive, Dropbox, GitHub, GitLab y Trello. Funciona de manera gráfica pudiendo arrastrar, conectar y alinear los componentes de los diagramas de forma sencilla.

Actualmente Draw.io se encuentra en proceso de transición a diagrams.net, este cambio no implica ningúna diferencia en el funcionamiento de la aplicación~\cite{diagramsNet}.



\subsection{Herramientas de gestión}

\subsubsection{GitHub}\label{github}
Github\footnote{\url{https://github.com}} se trata de una plataforma en la que es posible alojar proyectos usando un control de versiones Git.

Se ha utilizado esta herramienta para alojar el repositorio del proyecto desarrollado, ademas de gestionar la planificación de las tareas mediante el uso de issues y milestones.

En el siguiente enlace se puede encontrar el repositorio del trabajo: \url{https://github.com/smi0010/TFG_Herramientas_Realidad_Aumentada}.

Esta herramienta ha sido escogida por tener experiencia trabajando con ella en asignatura previas del grado.

\subsubsection{ZenHub}\label{ZenHub}
ZenHub\footnote{\url{https://www.zenhub.com/}} se trata de una herramienta para la gestión de proyectos. Se encuentra disponible para GitHub mediante una extensión del navegador.

ZenHub permite usar un tablero Kanban para el seguimiento y organización del proyecto. En el se representan las issues como tarjetas, que las podemos organizar en diferentes columnas.


\subsubsection{GitHub Desktop}
 GitHub Desktop\footnote{\url{https://desktop.github.com/}}, se trata de un software  que facilita el uso del sistema Git mediante una interfaz gráfica, permitiendo de esta forma uso más sencillo y intuitivo. Permite sincronizarse con los repositorios de GitHub, y escoger cual usar en cada momento de una manera  rápida y sencilla.
 
 
 Actualmente se encuentra disponible para sistemas Windows y masOS.

\subsection{Herramientas de desarrollo}

\subsubsection{Unity}
Unity\footnote{\url{https://unity.com/es}}, es un motor gráfico de desarrollo de videojuegos creado por Unity technologies, disponible para Windows, Linux y Mac OS. En él, se pueden desarrollar aplicaciones para distintas plataformas como, Windows, MacOS, Linux, Andorid, iOS, videoconsolas, WebGL, tvOS y Facebook.

\subsubsection{VisualStudio}
ViusalStudio\footnote{\url{https://visualstudio.microsoft.com/es/}}, se trata de un entorno de desarrollo para Windows, Linux y macOS. Es compatible con múltiples lenguajes de programación. C\# es uno de los principales lenguajes con los que es compatible, y dado que los scripts de Unity son en dicho lenguaje, se ha escogido visual studio para su desarrollo.


\subsubsection{Vuforia}
Como se ha explicado en el apartado \ref{sub:Def_Vuforia}, Vuforia\footnote{\url{https://developer.vuforia.com/}} se trata de un SDK para aplicaciones de AR, para las plataformas de Android, iOS, Windows, Unity y HoloLens. Proporciona las herramientas necesarias para desarrollar realidad aumentada en el entorno Unity.

\subsubsection{CoSpace}
Como se expuso en el apartado \ref{sub:Def_Vuforia}, CoSpaces se trata de una plataforma de apoyo educativo. Se puede acceder a ella tanto por web, como desde la aplicación de móvil. Permite crear entornos virtuales y de realidad aumentada para el MergeCube, desde su entorno de desarrollo web.

\subsubsection{Gimp}
Gimp\footnote{\url{https://www.gimp.org/}}, se trata de un software libre y gratuito, de edición de imágenes digitales. Se ha usado para la creación y edición de algunos de los sprites usados en la aplicación.

\subsubsection{Pencil}
Pencil\footnote{\url{https://pencil.evolus.vn/}} se trata de un software gratuito para el diseño de prototipos de interfaces gráficas para aplicaciones. Esta pensada para el diseño de interfaces tanto para dispositivos móviles como aplicaciones de escritorio.

\section{Técnicas utilizadas en el desarrollo del trabajo}
\subsubsection{Scrum}
Scrum se trata de una metodología de desarrollo ágil de software que se encuentra basada en el manifiesto ágil~\footnote{\url{https://agilemanifesto.org/principles.html}}. Se ha seguido esta metodología realizando srpints con una duración normalmente de una semana, y una reunión semanal al finalizar cada sprint en las que se mostraban los avances realizados y se planteaba la planificación del siguiente.

Para facilitar el seguimiento de esta metodología se ha utilizado en Github(\ref{github}) las issues y la extensión de ZenHub(\ref{ZenHub}), que permite organizar de una forma mas cómoda mediante tableros kanban\footnote{\url{https://es.wikipedia.org/wiki/Kanban_(desarrollo)}}.


REVISADO HASTA AQUI DE MOMENTO; AHORA ESTOY SIGO TRABAJANDO EN EL RESTO DE APARTADOS

