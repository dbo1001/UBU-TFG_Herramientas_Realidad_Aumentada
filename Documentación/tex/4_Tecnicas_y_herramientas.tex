\capitulo{4}{Técnicas y herramientas}


\section{Técnicas de desarrollo}

\section{Herramientas de documentación}
\subsection{LaTex}
Se trata de un sistema de composición de textos, con el objetivo de la creación de documentos de alta calidad tipográfica.
\section{Herramientas de gestión}
\subsection{GitHub}
Se trata de una plataforma donde se pueden guardar con un sistema de control de versiones los proyectos....

 Link del repositorio: \url{https://github.com/smi0010/TFG_Herramientas_Realidad_Aumentada}

\section{Herramientas de desarrollo}
\subsection{Unity}
Unity\footnote{\url{https://unity.com/es}}, es un motor gráfico de desarrollo de videojuegos creado por Unity technologies, disponible para Windows,Linux y Mac OS. En el, se pueden desarrollar aplicaciones para distintas plataformas como, Windows, MacOS,Linux, Andorid, iOS, videoconsolas, WebGL, tvOS y Facebook.
\subsection{VisualStudio}
ViusalStudio \footnote{\url{https://visualstudio.microsoft.com/es/}}, se trata de un entorno de desarrollo para Windows, Linux y macOS. Es compatible con múltiples lenguajes de programación. C\# es uno de los principales lenguajes con los que es compatible, y dado que los scripts de Unity son en dicho lenguaje, se ha escogido visual studio para su desarrollo.

\subsection{Vuforia}
Como se ha explicado en el apartado \ref{sub:Def_Vuforia}, Vuforia\footnote{\url{https://developer.vuforia.com/}} se trata de un SDK para aplicaciones de AR, para las plataformas de Android, iOS, Windows, Unity y HoloLens. Proporciona las herramientas necesarias para desarrollar realidad aumentada en el entorno Unity.
\subsection{CoSpace}
Como se exponía en el apartado \ref{sub:Def_Vuforia}, CoSpaces\footnote{\url{https://cospaces.io/edu/}} se trata de  es una plataforma de apoyo educativo. Se puede acceder a ella tanto por web, como desde la aplicación de móvil. Permite crear entornos virtuales y de realidad aumentada para el MergeCube, desde su entorno de desarrollo web.


