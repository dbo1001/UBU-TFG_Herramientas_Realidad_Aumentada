\apendice{Especificación de Requisitos}

\section{Introducción}
En el presente anexo se detallan los objetivos generales de la aplicación desarrollada y el catálogo de requisitos funcionales y no funcionales así como los diagramas de casos de uso.
\section{Objetivos generales}
\begin{itemize}
	\item Un objetivo es el de estudiar el estado del arte en este tipo de aplicaciones de realidad aumentada. 
	\item Realizar una aplicación para el ámbito educativo, empleando una de las aplicaciones AR estudiadas.
	\item Utilizar MergeCube para el diseño e implementación de la aplicación.
	\item Como ejercicio se ha diseñado el crecimiento de una planta, en el que se representaría las principales fases del crecimiento. Sería para usar en clases de biología de primaria/secundaria.
	\item Que pueda usarse en el mayor número de dispositivos compatible posibles.
\end{itemize}
\section{Catálogo de requisitos}
\subsection{Requisitos Funcionales}
 \begin{itemize}
	\item \textbf{RF 1:} Crear una aplicación educativa.
	\begin{itemize}
	\item \textbf{RF 1.1:} Permitir que el usuario configure el nivel.
	\item \textbf{RF 1.2:} Ofrecer ayuda o tutorial.	
	\item \textbf{RF 1.3:} Realizar capturas de pantalla y guardarlas en el dispositivo.
	\item \textbf{RF 1.4:} Permitir realizar encuestas de valoración.
	\item \textbf{RF 1.5:} Acceder a la información de los sensores del dispositivo.
	
	\end{itemize} 
	\item \textbf{RF 2:} Modelar un ejercicio de realidad aumentada sobre el crecimiento de una planta.
	\begin{itemize}
	\item \textbf{RF 2.1:} Representar visualmente el avance en cada fase.
	\item \textbf{RF 2.2:} Representar visualmente las herramientas con las que se avanza en el juego, simulando que el dispositivo puede manejar esa herramienta.
	\item \textbf{RF 2.3:} Cada fase de un nivel, estará representado por una etapa del crecimiento de una planta.
	
	\end{itemize} 
	
	
\end{itemize}
\subsection{Requisitos no Funcionales}
\begin{itemize}
	\item \textbf{RNF 1:} La aplicación debe de resultar fácil de usar y intuitiva para el usuario.
	\item \textbf{RNF 2:} Debe tener un rendimiento aceptable, para un uso cómodo.
	\item \textbf{RNF 3:} Debe de ser responsiva, pudiéndose adaptar la interfaz, a la pantalla de cualquier dispositivo-teléfono móvil.
\end{itemize}
\section{Especificación de requisitos}

\imagen{Anexos/casosdeuso}{Diagrama de casos de uso.}
\subsection{Descripción de los casos de uso}
%\tablaCasosDeUso{Caso de Uso 1: Menu}{}{}{}
\subsubsection{Caso de uso 1: Configurar nivel.}
\begin{itemize}
	\item \textbf{Descripción:} Configurar los aspectos básicos del juego en la aplicación.
	\item \textbf{Precondiciones:} Encontrarse en el menú principal.
	\item \textbf{Requisitos:} RF-1 RF-1.1
	\item \textbf{Acciones:}
	\begin{enumerate}
		\item Seleccionar en el menú el botón Opciones.
		\item Cambiar a la pantalla del menú de opciones.		
		\item Escoger una de las opciones a configurar.		
	\end{enumerate}
	\item \textbf{PostCondiciones:} Se ha entrado en la configuración correctamente. al salir los cambios se han guardado y hecho efectivos.
	\item \textbf{Excepciones:} -
\end{itemize}
\subsubsection{Caso de uso 1.1: Escoger modo target.}
\begin{itemize}
	\item \textbf{Descripción:} Escoger el tipo de target que se usará en la aplicación.
	\item \textbf{Precondiciones:} haber entrado en opciones
	\item \textbf{Requisitos:} RF-1 RF-1.1
	\item \textbf{Acciones:}
	\begin{enumerate}
		\item Elegir una de las opciones disponibles.
		\item Configurar parámetros a decisión del usuario.
	\end{enumerate}
	\item \textbf{PostCondiciones:} guardar los cambios realizados.
	\item \textbf{Excepciones:} -
\end{itemize}
\subsubsection{Caso de uso 2:  Seleccionar modo juego AR}
 \begin{itemize}
 	\item \textbf{Descripción:} Seleccionar uno de los modos de juego disponibles.
 	\item \textbf{Precondiciones:} Encontrarnos en el menú principal.
 	\item \textbf{Requisitos:} RF-1
 	\item \textbf{Acciones:}
 	\begin{enumerate}
 		\item Seleccionar en el menú el botón Modos de Juego.
 		\item cambiar a la pantalla del menú de modos de juego.
 	\end{enumerate}
 	\item \textbf{PostCondiciones:} Acceder al menú de modos de juego.
 	\item \textbf{Excepciones:}	-
 \end{itemize}
\subsubsection{Caso de uso 2.1: Seleccionar semilla.}
\begin{itemize}
	\item \textbf{Descripción:} El usuario escogerá una de las semilla/planta disponibles y se iniciara el modo AR de esa semilla.
	\item \textbf{Precondiciones:} Haber seleccionado el modo de las plantas.
	\item \textbf{Requisitos:} RF-1 RF-1.5 RF-2
	\item \textbf{Acciones:}
	\begin{enumerate}
		\item Seleccionar en una de las disponibles.
		\item Iniciar el modo AR sobre la semilla seleccionada.
	\end{enumerate}
	\item \textbf{PostCondiciones:} Se ha iniciado el modo de realidad aumentada correctamente.
	\item \textbf{Excepciones:} Si se trata de iniciar una planta aun no implementada, no se iniciará el modo AR.
\end{itemize}
\subsubsection{Caso de uso 2.1.1: Escoger una herramienta}
\begin{itemize}
	\item \textbf{Descripción:} Escoger entre una de las diferentes herramientas <<jugables>> del modo AR.
	\item \textbf{Precondiciones:}
	\begin{enumerate}
		\item Se ha iniciado el modo AR.
		\item Sensor de la cámara encendido.
	\end{enumerate}
	\item \textbf{Requisitos:} RF-2 RF-2.2
	\item \textbf{Acciones:} 
	\begin{enumerate}
		\item Pulsar  botón de la esquina inferior derecha.
		\item Se desplegará una barra con botones de distintas herramientas.
		\item Escoger una de las herramientas y pulsar.
		\item Usar la harramienta seleccionada y completar sus funciones en el nivel.
	\end{enumerate}
	\item \textbf{PostCondiciones:} La herramienta sale en pantalla y es funcional.
	\item \textbf{Excepciones:} 
	\begin{enumerate}
		\item La funcionalidad de la herramienta ya ha sido completada.
		\item El nivel ya ha sido completado.
	\end{enumerate}

\end{itemize}
\subsubsection{Caso de uso 2.1.2: Captura de pantalla. }
 \begin{itemize}
 	\item \textbf{Descripción:} Realiza una captura de pantalla de ese instante y la guarda en la galería del terminal.
 	\item \textbf{Precondiciones:}
	 		\begin{enumerate}
	 		\item Se ha iniciado el modo AR
	 		\item Sensor de la cámara encendido.
	 		
	 	\end{enumerate}
 	\item \textbf{Requisitos:} RF-1.3 RF-2 RF-2.1
 	\item \textbf{Acciones:}
	 	\begin{enumerate}
	 		\item Seleccionar captura de pantalla.
	 		\item Seleccionar captura de pantalla.
	 		\item Guardar en un archivo png.
	 		\item Volver al funcionamiento
	 	\end{enumerate}
 	
 	\item \textbf{PostCondiciones:}
	 	\begin{enumerate}
	 		\item Captura guardada correctamente como un archivo de imagen.
	 		\item El modo AR sigue funcionando y no se interrumpe.
	 		
	 	\end{enumerate}
 
 	\item \textbf{Excepciones:}  No hay espacio suficiente de almacenamiento.
 \end{itemize}
\subsubsection{Caso de uso  2.1.3: Visualizar interfaz.}
\begin{itemize}
	\item \textbf{Descripción:} La interfaz del modo AR se hace visible para el usuario.
	\item \textbf{Precondiciones:} Entrar en el modo jugable AR
	\item \textbf{Requisitos:} RF-2 RF-2.1 RF-2.2 RF-2.3
	\item \textbf{Acciones:}
		\begin{enumerate}
			\item Entrar en el modo AR.
			\item Cargar interfaz
		\end{enumerate}
	\item \textbf{PostCondiciones:}
		\begin{enumerate}
			\item La interfaz ha sido cargada correctamente.
		\end{enumerate}
	\item \textbf{Excepciones:} Error al cargar el modo AR.
\end{itemize}
\subsubsection{Caso de uso 3: Ayuda.}
\begin{itemize}
	\item \textbf{Descripción:} Proporciona una guía o ayuda básica a los usuarios, respecto al uso de la aplicación.
	\item \textbf{Precondiciones:} -
	\item \textbf{Requisitos:} RF-1 RF-1.2
	\item \textbf{Acciones:}
	\begin{enumerate}
		\item Seleccionar en el menú el botón Ayuda.
		\item Cambiar a la pantalla del menú de Ayuda.
	\end{enumerate}
	\item \textbf{PostCondiciones:} La ventana que muestra la información de ayuda, ha cargado correctamente.
	\item \textbf{Excepciones:} -
\end{itemize}
\subsubsection{Caso de uso 4: Encuestas}
\begin{itemize}
	\item \textbf{Descripción:}  Al completarse un nivel se da la posibilidad de realizar una encuesta sobre la app.
	\item \textbf{Precondiciones:} Completar un nivel.
	\item \textbf{Requisitos:} RF-1 RF-1.4
	\item \textbf{Acciones:}
	\begin{enumerate}
		\item Finalizar un nivel.
		\item Entrar en link de la encuesta.
		\item Redirigir a la página, usando el navegador del dispositivo.
		\item Contestar las preguntas.
	\end{enumerate}
	\item \textbf{PostCondiciones:} Se ha redirigido a la pagina de la encuesta..
	\item \textbf{Excepciones:} No hay conexión a Internet, por lo que no se puede acceder a la encuesta.
	
\end{itemize}

\subsection{Actores}
\begin{itemize}
	\item \textbf{Usuario:} Usuario que hará uso aplicación.
\end{itemize}


