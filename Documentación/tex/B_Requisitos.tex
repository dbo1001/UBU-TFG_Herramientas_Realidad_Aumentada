\apendice{Especificación de Requisitos}

\section{Introducción}
En el siguiente apartado se especificarán los requisitos y objetivos del ejemplo que se ha desarrollado.
\section{Objetivos generales}
 \begin{itemize}
	\item El objetivo principal es realizar una aplicación móvil de realidad aumentada, desarrollándola con algunas de las herramientas de realidad aumenta vistas.
	\item Realizar la aplicación, teniendo como principal marcador el MergeCube.
	\item Poder ser una aplicación de ámbito educativo.
	\item Se propone simular el crecimiento de una planta, pudiendo ver las diferentes etapas del desarrollo de la planta.
	\item Que pueda usarse en el mayor número de dispositivos compatible posibles.  
\end{itemize}
\section{Catalogo de requisitos}
\subsection{Requisitos Funcionales}
 \begin{itemize}
	\item \textbf{RF 1:} Menú en el que configurar diferentes modos.
	\begin{itemize}
	\item \textbf{RF 1.1:} Seleccionar entre diferentes modos de juego.
	\item \textbf{RF 1.2:} Ajustar tiempos de cada fase,  .
	\item \textbf{RF 1.3:} Seleccionar posibles tipos de plantas.	
	\end{itemize} 
	\item \textbf{RF 2:} Botones para seleccionar las diferentes herramientas (regar, abonar, etc ).
	\item \textbf{RF 3:} Botón para volver al menú de selección.
	\item \textbf{RF 4:} Una interfaz que indique el progreso de cada fase del juego.
	\item \textbf{RF 5:} Uso del giroscopio del dispositivo, para simular tener gravedad correctamente orientada.
	\item \textbf{RF 6:} Posibilidad de mover el dispositivo, alrededor del objeto representado por realidad aumentada.
	\item \textbf{RF 7:} Ofrecer información educativa implementada dentro del juego.
	\item \textbf{RF 8:} Ofrecer una Ayuda de usuario o tutorial.
	\item \textbf{RF 9:} Poder cambiar de idioma.
	\item \textbf{RF 10:} Poder hacer capturas de pantalla.
\end{itemize}
\subsection{Requisitos no Funcionales}
\begin{itemize}
	\item \textbf{RNF 1:} La aplicación debe de resultar fácil de usar y intuitiva para el usuario.
	\item \textbf{RNF 2:} Debe tener un rendimiento aceptable, para un uso cómodo.
	\item \textbf{RNF 3:} Debe de ser responsiva, pudiéndose adaptar la interfaz, a la pantalla de cualquier dispositivo.
\end{itemize}
\section{Especificación de requisitos}


