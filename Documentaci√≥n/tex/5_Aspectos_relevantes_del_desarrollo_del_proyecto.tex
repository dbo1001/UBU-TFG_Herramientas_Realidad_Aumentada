\capitulo{5}{Aspectos relevantes del desarrollo del proyecto}
En este apartado se explicarán los aspectos mas relevantes del desarrollo del proyecto. Las decisiones que se han tomado a lo largo del proyecto, los problemas que han surgido.

\subsection{Formación}
Para este proyecto ha sido necesario formarse en Unity, pues aunque contaba con conocimientos básicos de antemano, era necesario reforzar los conocimientos en cara al proyecto. Para ello se han seguido tanto videotutoriales como la documentación oficial de Unity y sus foros.
Para el desarrollo en Unity ha sido necesario revisar tutoriales respecto algunas de sus configuraciones.
Aunque contaba con la ventaja de que el lenguaje C\# usado por los scripts de Unity ya lo conocía, lo cual ha facilitado el proceso de desarrollo en Unity en esta parte. También el hecho de contar con un buen manual de usuario, y una comunidad de usuarios notable.

\subsection{Estudio herramientas}
Uno de los principales puntos de este proyectos ha sido el estudio de diferentes herramientas de realidad aumentada. Para la selección de estas herramientas se siguieron varios pasos. Primero dos de las herramientas de las cuales ya se tenían algunos conocimientos debido a la importancia de las compañías ARCore de Google y ARkit de Apple. Para el resto de herramientas se procedió ha buscar las herramientas que actualmente son mas populares y aquellas que destaquen por sus funcionalidades.

\subsection{Elección Herramienta de realidad aumentada}

Uno de los objetivos del proyecto era el de realizar un aplicación en la que poder poner aprueba las funcionalidades de la realidad aumenta. Para ello se debió de escoger entre una de las herramientas estudiadas. para esta selección se han tenido varios aspectos en cuenta: 
\begin{itemize}
	\item El primero la capacidad de esa herramienta para funcionar en el mayor número de dispositivos posibles.
	\item Que se pueda desarrollar una aplicación de forma gratuita.
	\item La continuidad de dicha herramienta, es decir que esa herramienta siga actualizándose y teniendo un soporte oficial.
	\item La comunidad que tiene. Con este aspecto me refiero a la cantidad de usuarios que usan la herramienta. Pues cuanto mas grande sea una comunidad de usuarios en un software, nos puede facilitar el resolver futuros problemas.
	
\end{itemize}

ArCore y Arkit, dada su tecnología eran una de las primeras opciones, pero en el caso de ArKit unicamente esta disponible para dispositivos iOS. Y en el caso de ArCore el número de dispositivos Android compatibles aún es algo limitado.
En el caso de las aplicaciones de MetaVerse, ZapWorks, WikiTude y 8thWall, se han descartado dado que para para el desarrollo de una aplicación unicamente ofrecen una versión gratuita por un tiempo limitado.

En el caso de OpenCv su principal problema es que en comparación con otras herramientas de realidad aumentada, OpenCv se queda limitada en cuanto a sus posibilidades.

Finalmente la decisión estaba entre Kudan y Vuforia. Con kudan que parece ser un software muy prometedor, el problema estaba en que el SDK disponible no fui capaz de funcionara, es probable que se deba a que lleve tiempo sin actualizarse, pues la ultima versión de Unity en la que esta testado era la 2018. sumado ha este problema esta también el que no tiene una comunidad muy grande ni una guía muy detallada de uso, por lo que buscar posibles soluciones a dudas o problemas se complicaría bastante.
Es por esto que finalmente me decante por Vuforia, pues  ofrece la posibilidad de desarrollar gratuitamente, siempre que no se publique con el objetivo comercial. Es compatible 


\subsection{Elección de Unity}
Para desarrollar un ejemplo que aplicara las herramientas estudiadas, se tuvo en cuenta aquellos entornos con los eran compatible, ademas de a ser posible buscar un entorno gratuito. Unity se trataba del que mejor concordaba con las características en la mayoría de herramientas. Ademas de que cuenta con um amplio manual de usuario, y un gran apoyo de la comunidad.

\subsection{Problemas Instalación Vuforia en Unity}
Para incorporar en el proyecto de Unity la herramienta de Vuforia, hay varios métodos, en este caso yo decidí hacerlo desde la Asset Store de Unity. Para poder instalarlo, simplemente debemos buscar en la tienda Vuforia. Entre los resultados podemos encontrar diferentes Assest que ofrece PTC, la compañía de Vuforia. Hasta aquí todo bien, pero en vez de seleccionar Vuforía Engine, que contiene unicamente el engine de Vuforia, seleccione y importe Vuforia Core Samples, que se trata del engine de Vuforia pero viene con múltiples ejemplos desarrollados de vuforia. De este hecho no percaté hasta bastante mas adelante. 

Me pude dar cuenta de esto por el peso de la aplicación y es que el APK era demasiado pesado para el contenido que estaba utilizando. A ese elevado peso de la APK, no le di demasiada importancia hasta que hice pruebas en un proyecto aparte, en los que solo instale Vuforia Engine y pude comprobar que en estos proyectos el peso era considerablemente menor.

Aparte del peso de la aplicación, no suponía ningún problema en el funcionamiento, pues realmente el engine era el mismo solo que en uno tenía mas recursos. Dado que a la larga el este excesivo peso aun que no fuera un problema, si sería un inconveniente. empece a revisar el proyecto para borrar los elementos que hacian que el peso de la aplicación ascendiera tanto. Finalmente si les puede encontrar en la carpeta \textbf{StreamingAsets}, \textbf{Edit}, y borrar todos los elementos excepto el que tendrá el nombre del prefab del MergeCube añadido.

\subsection{Corrupción del proyecto}

En dos ocasiones el proyecto de Unity se corrompió, de forma que la única opción fue restaurarlo con una copia de seguridad. 

En ambas ocasiones el problema fue el mismo, todos los scripts pasaban a ser inutilizables en el proyecto. Esto impedia la posibilidad de asociar un script a un GameObject para que este usará las funciones del script. Este problema no solo impedía realizar nuevas asociaciones, sino todos los objetos que ya estaban asociados a algún script ya no lo estaban.

La lamentablemente no puede encontrar el origen del problema, ni una solución. Algunas de las posibles soluciones que encontré en foros, eran borrar los ficheros temporales o reimportar el proyecto. La lamentablemente ninguno funciono y tuve que restaurarlo con un copia. 




