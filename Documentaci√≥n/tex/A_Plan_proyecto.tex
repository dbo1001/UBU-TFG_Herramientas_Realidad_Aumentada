\apendice{Plan de Proyecto Software}

\section{Introducción}
 Para la organización del trabajo se ha utilizado GitHub\footnote{\url{https://github.com}} con la extensión de ZenHub\footnote{\url{https://www.zenhub.com/}} para facilitar el seguimiento de las issues gracias a las funcionalidades que ofrece la extensión.
 
En el siguiente enlace se encuentra el repositorio del trabajo: \url{https://github.com/smi0010/TFG_Herramientas_Realidad_Aumentada}.
\section{Planificación temporal}

En un principio, se ha llevado a cabo una reunión a la semana para cada sprint, en la que se exponían los avances realizados durante el sprint de la semana y la planificación del siguiente.

A continuación se explicarán un resumen de los sprints que se han ido desarrollando durante el proyecto.

\subsection{Sprint 0 (15/01/2020-22/01/2020)}
Durante la primera reunión, estuvimos hablando sobre el objetivo del proyecto, de que visión teníamos sobre él, y de hacia donde le podríamos dirigir.
Las tareas que se hicieron fueron sobre la creación del repositorio y la instalación de \LaTeX{} para la documentación. También comencé a investigar sobre diferentes herramientas de realidad aumentada de forma general, para poder compararlas y ver las ventajas y desventajas que tienen.

\subsection{Sprint 1 (22/01/2020-29/01/2020)}
Durante este sprint, hemos acordado ir mejorando y completando  la investigación respecto las herramientas de realidad aumentada que tenía seleccionadas.
Para este sprint decidí centrarme en las herramientas Vuforia y Mergecube.
También como uno de los objetivos del TFG es trabajar con Unity para la creación de ejemplos/proyectos, he incluido una pequeña introducción sobre Unity y los pasos necesarios para su instalación.
\subsection{Sprint 2 (29/01/2020 - 5/02/2020)}
Durante el segundo sprint me centré en documentar sobre las diferentes técnicas de realidad aumentada que se utilizan por la mayoría de herramientas.
También en documentar las herramientas de ArCore y ArKit.

Otro de los puntos que tratamos, fue sobre posibles ideas para la elaboración de un ejemplo en el que poner aprueba las herramientas AR. Surgió la idea de realizar una especie de juego o aplicación <<educativa>> que pueda servir como inicio a la programación. Aunque esta idea nos dimos cuenta que depende de cómo la enfoquemos puede ser demasiado compleja y amplia como para hacer un ejemplo sencillo.
\subsection{Sprint 3 (5/02/2020 - 12/02/2020)}

Me centré en la documentación de las herramientas de Kudan y 8thwall.
Debido que estuve enfermo gran parte de las semana no pude avanzar mas.
\subsection{Sprint 4 (12/02/2020 - 26/02/2020)}
Durante este sript decidí centrarme en las herramientas de Wikitude, OpenCV, ZapWorks, y en realizar también correcciones y mejoras de la documentación que el tutor había señalado.
También estuve pensando en posibles ideas para realizar un ejemplo aplicando las herramientas de realidad aumentada. 
\subsection{Sprint 5 (26/02/2020 - 04/03/2020)}
Se estuvo hablando respecto al ejemplo, para empezar ya con el. Sobre la idea de hacer un pequeño juego en el que hay un <<laberinto>> y que el usuario tuviera que resolver el camino moviéndose por comandos de una forma similar a una programación por blocking pero simplificado, todo esto aplicado con las herramientas de realidad aumentada. 

Dado que aún la idea parecía algo incompleta y complicada de ajustar a las herramientas de realidad aumentada, vamos a intentar primero un ejemplo mas sencillo, este sería uno en la que poder ver las fases de crecimiento en una <<planta>>. 
Durante el sprint probare a poder usar la cámara de forma que sea como una regadera, así cuando este encima del Mercube donde estaría la planta virtual, se simule en realidad aumentada, que cae agua desde la cámara del teléfono hacia la planta. Otra forma pensada mas simplificada, sería simplemente añadir en pantalla un botón, que al presionarle se simule que cae agua sobre la planta sin tener que mover el teléfono.

\subsection{Sprint 6 (04/03/2020 - 11/03/2020)}

Durante la reunión de este sprint mostré los primeros pasos que había realizado del ejemplo a desarrollar en Unity para aplicar herramientas AR. También comenté mis ideas sobre como continuar y los puntos en que me quería centrar durante este sprint.

Centrarme en poder detectar las partículas que simulan ser agua, cuando estas colisionan con un objeto determinado. Y que contando el número de contactos poder determinar la progresión para poder pasar a otra etapa o nivel. También hacer uso del sensor giroscopio del móvil, para determinar con la inclinación del dispositivo la cantidad de partículas que caen.

Como la idea del ejemplo sería seguir el crecimiento de una planta en realidad aumentada desde un punto educativo, pensamos en que aparte de poder echarle agua, también se podría cambiar entre diferentes opciones, como abonos, sulfatos, etc. También se podrían considerar establecer normas del tipo, escoger la temporada en la que se planta, si necesita que le quiten malas hiervas, etc. Por lo que habría que incluir una forma de cambiar entre esas diferentes opciones.

Otro de los aspectos que he estado mirando, es mejorar el posicionamiento virtual del objeto 3D en el marcador. Ya que con la primera configuración que he probado, cuando se mueve la cámara (el móvil) el objeto se mueve de su posición demasiado y no queda estable.

\subsection{Sprint 7 (11/03/2020 - 18/03/2020)}
Durante la reunión de este sprint mostré los avances que había realizado sobre el ejemplo de la aplicación. También les expliqué que estuve realizando varias pruebas añadiendo las funcionalidades del SDK de MergeCube en la aplicación. Consiguiendo mejorar los resultados con el uso conjunto de Vuforia + Mergecube, que únicamente con Vuforia.

Otro de los aspectos tratados fue sobre la herramienta de CoSpace, la necesidad de explicar las ventajas de su uso, para explicar su viabilidad económica.

Durante la semana realicé algunos avances en la documentación, y estuve solucionando un bug en el uso del giroscopio. El bug en cuestión se traba de que el eje de gravedad de las partículas, no coincidía con el eje de gravedad normal. El problema resultó ser que, el parámetro <<Simulation Space>> debía de estar con la opción Word. Otro problema de este bug, era que el eje Z, no tenía la interpretación deseada para nuestro ejemplo. Para esto en el script del giroscopio, se modificó para invertir el eje Z.
 
\subsection{Sprint 8 (18/03/2020 - 25/03/2020)}
En este Sprint, la reunión se tuvo que realizar por videoconferencia, debido a la cuarentena en la que se encontraba el país durante esos días. Expliqué los avances que había realizado en la documentación y algunas de las dudas que me habían surgido sobre como debería avanzar en ciertos puntos de la documentación. También expliqué sobre el bug que había corregido en la aplicación, aunque no puede mostrárselo ya que al ser la reunión online  no pude preparar un vídeo para enseñarlo.
 
Para este sprint seguí avanzando en la documentación, me centré en especificar los requisitos y objetivos de la aplicación. También diseñé en unas imágenes un prototipo de la interfaz. En la aplicación trabajé en añadir diferentes fases del crecimiento de un planta, que cuando se completa el progreso de una fase salta a la siguiente. 
\subsection{Sprint 9 (25/03/2020 - 01/04/2020)}
Durante el sprint de esta semana me he centrado en dotar a la aplicación de un prototipo de la interfaz y de menú funcional. En la interfaz el principal punto en que me he centrado, ha sido el de realizar una barra de progreso, la cual se rellena a medida que el nivel es completado. En el menú, me he centrado entre la navegación entre los diferentes niveles del menú, y el más importante, que desde el menú se pueda inicializar el juego correctamente.

También se ha investigado sobre la posibilidad de que la aplicación mande un fichero o un mensaje, a un servidor o página web. Esto serviría para realizar un encuesta de opinión de la aplicación sin tener que salir de ella. 
\subsection{Sprint 10 (01/04/2020 - 15/04/2020)}
Como objetivo durante el sprint tenía mejorar la interfaz, principalmente con la opción de poder activar y desactivar las herramientas del juego y añadir ventanas emergentes en las que se podría leer información de ayuda para el usuario. También centrarme en mejorar algunos de los apartados en la documentación.
Otro de los puntos era el hacer pruebas en CoSpaces, intentando realizar un ejemplo similar al que se está desarrollando en Unity y poder hacer comparaciones.
\subsection{Sprint 11 (15/04/2020 - 22/04/2020)}
Durante este sprint me propuse por objetivos, implementar una nueva herramienta que simule una nueva función para el crecimiento de plantas sumado al de regar ya implementado. En concreto trataría de representar el uso de un tipo de fertilizante. Para completar esto, han surgido también los objetivos de completar las condiciones por las que se considera una fase completada y se puede iniciar la siguiente fase, ahora por ejemplo para completar una fase aparte de recibir cierta cantidad de agua, también debería de haber recibido fertilizante. Y también el objetivo de actualizar la interfaz para que puede representar la nueva funcionalidad, la idea sería que en la barra de progreso cuando se utilice el fertilizante, quede reflejado de alguna forma, para que así el usuario vea que se ha usado.
\subsection{Sprint 12 (22/04/2020 -29/04/2020)}
Para este sprint durante la reunión mostré los avances sobre la aplicación. Mientras explicaba los avances, expliqué también algunos pequeños bugs que me estaba encontrando.
Los objetivos para este sprint son los bugs que se han encontrado, el primero se trata de un bug en la animación de la herramienta, el cual consiste que al terminar dicha animación el objeto se queda en la posición final, la herramienta debería volver a su posición de origen para poder usarla nuevamente. Otro de los bugs se trata de que no se distinguía correctamente si un sistema tiene el sensor del giroscopio, por lo que, en caso de no tenerlo, no se configura correctamente y las partículas que se deberían ver afectadas por el giroscopio se quedan flotando. Y el otro bug se trata de uno de adaptabilidad, pues un elemento en las pruebas del entorno de desarrollo en el ordenador se ve en su posición correcta, pero al probar la aplicación en un dispositivo móvil, ese elemento no aparece en la posición correcta en pantalla y debería de poder adaptarse a cualquier tipo de pantalla. 

\subsection{Sprint 13 (29/04/2020 - 06/05/2020)}
Los objetivos de este Sprint serán centrarse en la documentación. Principalmente he decidido centrarme en los apartados de la introducción de la memoria, el de Técnicas y Herramientas explicando sobre la experiencia de usar CoSpaces, y actualizar el apartado de los objetivos del proyecto.

Centrados en la aplicación que se esta desarrollando, se pondrán por objetivos seguir investigando posibles soluciones al bug de adaptabilidad que surge al ejecutar la aplicación en un dispositivo móvil. También añadir una nueva condición para que cada fase tenga un tiempo limite para completarla. Para ello hay que implementar un marcador que muestre el tiempo que queda y al finalizar se muestre algún mensaje para que el usuario se percate de que se ha quedado sin tiempo.

\subsection{Sprint 14 (06/05/2020 - 13/05/2020)}
Para el sprint de esta semana me puse por objetivos el crear un menú en realidad aumentada, de forma que se pueda cambiar entre un menú normal y el de realidad aumenta, ademas de añadir imágenes que indiquen el nombre de cada parte de la planta. También la investigación para intentar corregir dos bugs, similares en los que los elementos en pantalla se posicionan incorrectamente dependiendo de la resolución de pantalla o si se ejecuta en un dispositivo móvil impidiendo que se vean de forma correcta. 
Otro punto a investigar es reducir el peso de la aplicación, pues actualmente pesa más de 80MB, y hay algunos archivos que no están en uso, por lo que habría que clasificar cuales se pueden eliminar.

Otro punto a investigar se trata de añadir alguna de las otras funciones de realidad aumentada que posee Vuforia, ver cuales de las que ofrece puede ser más apropiada para complementar el ejemplo desarrollado.

\subsection{Sprint 15 (13/05/2020 - 20/05/2020)}
 Durante la reunión mostré la implementación del menú Ar en la aplicación que había realizado en el anterior sprint. También estuve explicando las conclusiones del estudio de posibles nuevas funciones de Vuforia a usar, destaque 2, Device tracking y Crear un target personalizado temporal, por lo que uno de los objetivos para el presente sprint será implementar estas funcionalidades en la aplicación. Acorde con el objetivo de añadir estas dos nuevas funcionalidades, habría que actualizar la interfaz. Otro de los objetivos para el sprint es crear un pequeña guía para que los nuevos usuarios puedan aprender los pasos básicos del funcionamiento de la aplicación.
 
 \subsection{Sprint 16 (20/05/2020 - 27/05/2020)}
 
 En la reunión del sprint mostré un vídeo de demostración de las funciones nuevas implementadas. Se me recomendó el subir este tipo de vídeos que pueda haciendo, y adjuntarlos en la documentación.
 
 Para este sprint el objetivo mas principal es darle un pulido a la aplicación en la interfaz y demás para que se acerque mas al lo que debería ser el producto final. Otra mejora en la aplicación será implementar  la posibilidad de configurar desde el menú que tipo de realidad aumentada usar y poder ajustar otros posibles parámetros.  También seguir completando y mejorando la documentación, para este sprint me centraré en el punto de la historia de la realidad aumentada y comenzaré ha construir la tabla comparativa de las herramientas.
 
 \subsection{Sprint 17 (27/05/2020 - 03/06/2020)}
 
 En la semana de este sprint hemos decido poner por objetivos, seguir mejorando la aplicación, mejorando la interfaz y mas aspectos, acercándole a lo que sería el diseño final, pues aún hay botones y otros elementos que no tienen uso, o que en algún momento su único propósito era de realizar pruebas de funcionalidades o diseños. Otro de los objetivos tratará de añadir mínimo un segundo modelo de una planta, para que la aplicación pueda mostrar mas modelos. Respecto al crecimiento de las plantas también se mejorará las fases, pues ahora mismo cada fase tiene las mismas condiciones, será recomendable que cada fase estuviese ajustada a lo que podrían ser sus condiciones reales.
 
 En la aplicación cuando en el menú se cambia entre menú normal y en realidad aumentada, el menú siempre vuelve a la página principal, se pone por objetivo, conseguir que cuando se cambie de tipo se conserve  la fase del menú. Es decir que si estoy en las opciones del menú, cuando cambie al modo de realidad aumentada siga en las opciones.
 Seguir mejorando los apartados de la documentación.
 
 \subsection{Sprint 18 (03/06/2020 - 10/06/2020)}
  
 Durante la reunión mostré los cambios de la interfaz del menú, y la mejora para que al cambiar entre menú normal y AR estén en la misma página. Al hacer esto me percate de un bug en el que en cierta página concreta no se cambia correctamente, con la consecuencia de que se superponen páginas. Este sera uno de los objetivos a resolver durante este sprint.
 
 Otro de los avances que mostré fueron los modelos de plantas que había añadido y como desde el menú se podía escoger cada una.  También expliqué que debido al modo en el que había realizado en primer momento el script que controla el cambio de fase de las plantas, no era posible añadir mas plantas de una forma eficiente. Por lo tanto mejorar este script, para que se pueda añadir tantos modelos sean necesarios de forma eficiente y funcional será otro de los objetivos del sprint.
 
Otro de los objetivos será el de completar la tabla comparativa añadida en el anterior sprint y seguir mejorando apartados de la documentación.
 
\subsection{Sprint 19 (10/06/2020 - 17/06/2020)}

 Para este sprint me centraré en dejar la aplicación en una versión cercana a la definitiva, para ello me concentraré en corregir los pequeños bugs que aún puedan quedar. Durante la reunión mientras mostraba los progresos de la semana anterior, nos dimos cuenta de dos bugs. El primero y mas importante, se trata de que el progreso de los niveles, no se reinicia al cambiar a otra planta, es decir en el caso de que se haya completado todas las fases o nos quedemos a la mitad, si volvemos al menú y escogemos otra planta o otro modo distinto, el progreso será el mismo de antes, por lo que no se podrá avanzar adecuadamente. Otro de los bugs vistos se trata de al finalizar el tiempo máximo del que se dispone para realizar las tareas, salta un mensaje de texto para informar de que se ha acabado el tiempo, este mensaje no se ve adecuadamente debido al encuadre y al tamaño de la fuente. También será un objetivo, el revisar la aplicación en busca de posibles bugs importantes como los indicados anteriormente para añadirlos, y resolverlos durante este sprint.
 
 El otro punto del sprint es centrarse en los apartados de los anexos y la memoria que están menos avanzados.
 
 \subsection{Sprint 20 (17/06/2020 - 24/06/2020)}
 Para este sprint el objetivo principal es seguir completando la memoria y los anexos. Durante la reunión mencioné en que apartados había estado trabajando, y las dudas que me habían surgido a la hora de completar algunos.
 
 Se ha hablado también de implementar la función de realizar capturas de pantalla y que se guarden en la galería, pues en un principio se había propuesto en los requisitos y aún no se había realizado. Por lo demás en la aplicación se seguirá revisando en busca de posibles bugs o pequeños aspectos que se puedan mejorar. 
 
  \subsection{Sprint 21 (24/06/2020 - 26/06/2020)}
 Para este sprint dado que se acercan la fecha de entrega, hemos decido acortar la duración del sprint para acelerar el avance.
 Concretamente en este sprint hemos acordado el centrarnos en los apartados de El estudio de viabilidad, Los Requisitos, y del Diseño de Anexos para poder darlos por cerrados. En la memoria nos centraremos en el apartado de Trabajos relacionados.
 
\subsection{Sprint 22 (26/06/2020 - 1/07/2020)}
Durante este sprint el principal objetivo es el de avanzar en la aplicación, terminando de implementar la captura de pantalla, en intentos previos me había encontrado con algún contratiempo. También rematar algunas características que no se habían completado totalmente, en el caso de el menú algunas de las pantallas se encuentran sin enlazar a su botón correspondiente para una correcta navegación.

 En los anexos el objetivo es centrase en el apartado del manual del programador.
  
\subsection{Sprint 23 (1/06/2020 - 3/07/2020)} 
En el anterior sprint uno de los objetivos era avanzar en el apartado del manual de programador, pero debido a complicaciones con otros objetivos no se ha pudo realizar. Por lo que ese será el primer objetivo para este sprint.
Siguiendo la dinámica, tratar de avanzar en otros apartados de la documentación.
En la aplicación teníamos previsto añadir una ayuda o guía para el usuario, para que sea más fácil su creación, se ha planteado la posibilidad de crearla en un html que se habrá y visualice desde la propia aplicación, por lo que otro objetivo trata de investigar de forma rápida, la posibilidad de hacerlo de este modo.
 
\subsection{Sprint 24 (3/07/2020 - 8/07/2020)} 
 Durante este sprint se procurara avanzar en los apartados del manual del usuario y a ser posible mas apartados de la memoria.
 Se ha encontrado un bug en la sincronización entre el menú normal y el de realidad aumentada, el bug en cuestión es que en el apartado de opciones, no se mantiene la opción seleccionada cuando se intercambia entre ambos modos del menú. El funcionamiento esperado es que independientemente de si esta manejando el menú en realidad aumentada o en el modo normal, ambos mantengan la misma configuración.
 
 \subsection{Sprint 25 (8/07/2020 - 10/07/2020)} 
 Para este sprint se procederá ha finalizar el modo de captura target de la aplicación. En dicho modo surgió el inconveniente que no se podía ``jugar` regando las plantas al igual que con  MergeCube, pues parece que estén en diferentes planos y nunca se puede hacer contacto. El objetivo será buscar una solución ha este problema, para que se pueda ``jugar`  con un target personalizado. 
 
 También se ha propuesto incluir un modo en el que se pueda hacer uso de la detección de múltiples marcadores simultáneos, pudiendo seleccionar y mostrar distintos modelos 3D, está propuesta puede ser viable pues en pruebas anteriores de captura target se había conseguido resultados similares, aun que sin probar a colocar modelos distintos en cada marcador.

 \subsection{Sprint 26 (10/07/2020 - 13/07/2020)} 
 Durante este sprint se revisaran las correcciones que los tutores han  indicado de la documentación que hay actualmente, para aplicarlas. También se procederá ha mejorar y completar mas apartados de la documentación.
 
 Respecto a la aplicación se rematará las mejoras que se realizaron durante el anterior sprint. En el modo juego usando captura target, debe ser mejorado para que al realizar una segunda captura sea posible seguir jugando correctamente. En el modo multitarget, se incluyo la posibilidad de cambiar el tamaño de un modelo, hay que añadir la posibilidad de elegir el modelo que se quiere cambiar de tamaño. 
 
\section{Estudio de viabilidad}

\subsection{Viabilidad económica}

En este apartado se evaluará la viabilidad económica de las herramientas que se han investigado.


CoSpaces, ofrece varias tarifas aparte de la básica que es gratuita, la más barata de 69,99~\euro{}  anuales y la mas cara es de 1349,99~\euro{} anuales.

La tarifa básica, ofrece la creación de una clase con capacidad para 30 personas (alumnos y profesores). En esta clase virtual, el profesor puede crear únicamente una tarea y asignarse al alumnado de forma individual o por grupos. El número máximo de profesores por clase en la tarifa básica es de uno.

La tarifa básica, permite la creación de un máximo de dos espacios, es decir un máximo de dos proyectos. En la creación de ejemplos, estos se encuentran limitados a unos pocos modelos 3D de CoSpace, y a un limitado número de opciones de programación por CoBlocks. En la tarifa básica no es posible realizar proyectos para el MergeCube.

Con la tarifa Pro de 69,99~\euro{}, el número de personas máximas para una clase, aumenta en 5, permitiendo también que haya mas de un profesor en la clase. Permite crear mas de una clase. En el apartado de espacios, la tarifa Pro, permite crear mas de dos espacios, y en estos espacios da acceso a todos los modelos 3D de CoSpace, y permite utilizar en la programación todos los bloques de CoBlocks y scripts.
Si se desea utilizar el Meregecube, la licencia asciende hasta los 79,99~\euro{}. 

El resto de tarifas que posee CoSpaces, su principal característica es el número de personas simultáneas para una clase que puede tener, subiendo en la última tarifa hasta 400 personas adicionales.

La tarifa que sería recomendable se trata de la primera ofertada en el plan Pro, ya que aunque se vaya a utilizar con un número de alumnos superior al ofertado, sería posible organizarse por turnos.

\subsubsection{Costes Personal}
El proyecto ha sido desarrollado por una única persona, durante un periodo de 6 meses a tiempo parcial. Se considerará un salario de 1300\euro.

Para calcular los porcentajes de cotización de la seguridad social, se tendrá en cuenta los datos de cotización 2020~\cite{contingencias2020}:

Por parte de la empresa:
\begin{itemize}
	\item Contingencias Comunes (Empresa): 23,6\%
	\item Formación Profesional (Empresa): 0,6\%
	\item Desempleo de tipo general (Empresa): 5,5\%
	\item Fondo de Garantía Salarial (Fogasa) 0.2\%
	\item Total 29,9\%:
\end{itemize}

Por parte del empleado:
\begin{itemize}
	\item Contingencias Comunes (Trabajador): 4,7\%	
	\item Formación Profesional (Trabajador): 0,1\%
	\item Desempleo de tipo general (Trabajador): 1,55\%
	\item Total 6,35\%
\end{itemize}

En la tabla \ref{tabla:costesPersonal} se utilizaran los supuestos costes de la empresa, para calcular el coste total.

El coste de personal total será de 10132,2\euro.

\tablaSmall{Costes Personal.}{l r }{costesPersonal}{\multicolumn{1}{l}{Concepto}& Coste (\euro)\\}{
	Salario Bruto& 1300~ \\
	Seguridad social(Empresa 29,9\%)& 388,7~  \\
	\midrule
	Coste mensual&1688,7~\\
	\midrule
	Coste Total&10132,2~\\
	}

\subsubsection{Costes Material}
Para el proyecto se ha requerido de algunos gastos en hardware y software.

\begin{itemize}
	\item Hardware:
	Se tendrá en cuenta el equipo con el se ha desarrollado la aplicación, un ordenador valorado en 1100~\euro. Una WebCam valorada en 70~\euro, para realizar pruebas en el ordenador. Y un smartphone valorado en 210~\euro. El Mergecube con un coste de 19,99~\euro. Los componentes se amortizarán en 4 años. En la tabla \ref{tabla:costesHardware} se pueden apreciar los costes  y las amortizaciones\footnote{Los costes de amortización son calculados a 6 meses, que es la duración del proyecto.} agrupados.
	\item Software: Algunas de las herramientas software que se han utilizado en el proyecto se traban de software gratuitos, o se han utilizado unicamente las pruebas o niveles gratuitos que ofrecen. Por el contrarío en algunos casos ha sido necesario, recurrir al pago. Uno de los casos es con CoSpaces, realizando una prueba de un mes, a un coste de 79,9~\euro/año. En la tabla \ref{tabla:costesSoftware} se pueden apreciar los costes agrupados.
\end{itemize}



Los costes totales de material, son de 417,48~\euro.


\tablaSmall{Costes hardware.}{l r r}{costesHardware}{\multicolumn{1}{l}{Concepto}& Coste( \euro) & Amortización( \euro)\\}{
	MergeCube& 19,99~&19,99 \\
	Ordenador& 1100~&137,5~  \\
	WebCam& 70~&8.75~  \\
	SmartPhone& 210~&26,25~  \\
	\midrule
	Total& 1399,99~&192,49~\\}

\tablaSmall{Costes software.}{l r }{costesSoftware}{\multicolumn{1}{l}{Concepto}& Coste ( \euro)\\}{
	Windows 10& 145~ \\
	CoSapaces& 79.99~/año  \\
	\midrule
	Total&224,99~\\}

\subsubsection{Costes Totales}

	El coste total durante el desarrollo, es de 10549,68~\euro. En la tabla \ref{tabla:costesTotales} se pueden apreciar el conjunto de todos los supuestos costes calculados.
	
\tablaSmall{Costes Totales}{l c  }{costesTotales}{\multicolumn{1}{l}{Concepto}& Coste ( \euro)\\}{
	Personal& 10132,~ \\
	Hardware& 192,49~  \\
	Software& 224,99~  \\
	\midrule
	Total& 10549,68~\euro\\}

\subsection{Viabilidad legal}

En el caso de las licencias del software usado, se ha aprovechado el uso gratuito. En el caso de Unity se trata de la licencia de uso personal~\footnote{\url{https://store.unity.com/es/compare-plans}}. En el caso de Vuforia, el desarrollo de aplicaciones es gratuito, pero si se tuviese la intención de publicarla con uso comercial sería necesario acogerse a una de las licencias de pago. En la tabla \ref{tabla:licenciasSoftware} se encuentran recogidas el tipo de licencias usadas.

El proyecto se han utilizado recursos para Unity de licencias gratuitas. Se tratan de algunos modelos 3D, imágenes, o Fonts que contaban con una licencia Creative Commons Atribution\footnote{\url{https://creativecommons.org/licenses/by/4.0/}} para su uso. En la tabla \ref{tabla:componentesUnity} se encunetran recogidos los distintos tipo de recursos usados en el proyectos Unity y sus licencias.

\tablaSmall{Licencias Software}{l c c }{licenciasSoftware}{\multicolumn{1}{l}{}& Versión& Licencia\\}{
	Unity& 2019.2.19f1& Unity Personal\\
	Vuforia& 8.6.10& Gratuita y Propietaria \\
	VisualStudio& 2017 &  Propietaria\\
	Gimp& 2.10& GNU\\
	}

\tablaSmall{Componentes usados en Unity}{l c c }{componentesUnity}{\multicolumn{1}{l}{}& Versión& Licencia\\}{
	Plants~\cite{unityAssetPlant}\footnote{\url{https://assetstore.unity.com/packages/3d/vegetation/plants/plants-150261}}& 1.0 & Extension Asset\\
	Cartoon Farm Crops~\cite{unityFarm}\footnote{\url{https://assetstore.unity.com/packages/3d/vegetation/plants/cartoon-farm-crops-79777}}	& 1.0& Extension Asset\\
	Garden Realistic Tools~\cite{unityGarden}\footnote{\url{https://assetstore.unity.com/packages/3d/garden-realistic-tools-68960}}
	& 1.0&Extension Assest\\
	Girasol~\cite{sketchfabGirasol}\footnote{\url{https://sketchfab.com/3d-models/girasol-9e3e900295254b82bc144b3ca4333b80}}
	& 1.0& Creative Commons Attribution\\
	Icono Fertilizante~\cite{iconoFertilizante}\footnote{\url{https://www.flaticon.es/icono-gratis/fertilizante_1465973?term=fertilizante&page=1&position=45}}&&Licencia de Flaticon
	\\
	Icono Gotas de lluvia~\cite{iconoGotasLluvia} &1 & Free for commercial use \\
	Dinomouse-Regular~\cite{fontDinomouse}& &Free For
	Commercial Use\\
	RacingSansOne-Regular~\cite{fontRacingSans}& &Free For
	Commercial Use\\
	}