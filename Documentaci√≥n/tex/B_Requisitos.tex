\apendice{Especificación de Requisitos}

\section{Introducción}
En el presente anexo se detallan los objetivos generales de la aplicación desarrollada y el catálogo de requisitos funcionales y no funcionales así como los diagramas de casos de uso.
\section{Objetivos generales}
\begin{itemize}
	\item Un objetivo es el de estudiar el estado del arte en este tipo de aplicaciones de realidad aumentada, realizando una aplicación para el ámbito educativo.
	\item Utilizar MergeCube para el diseño e implementación de la aplicación.
	\item Como ejercicio se ha diseñado el crecimiento de una planta, en el que se representaría las principales fases del crecimiento. Sería para usar en clases de biología de primaria/secundaria.
	\item Que pueda usarse en el mayor número de dispositivos compatible posibles.
\end{itemize}
\section{Catálogo de requisitos}
\subsection{Requisitos Funcionales}
 \begin{itemize}
	\item \textbf{RF 1:} Menú en el que configurar diferentes modos.
	\begin{itemize}
	\item \textbf{RF 1.1:} Seleccionar entre diferentes modos de juego.
	\item \textbf{RF 1.2:} Ajustar tiempos de cada fase,  .
	\item \textbf{RF 1.3:} Seleccionar posibles tipos de plantas.	
	\end{itemize} 
	\item \textbf{RF 2:} Botones para seleccionar las diferentes herramientas (regar, abonar, etc ).
	\item \textbf{RF 3:} Botón para volver al menú de selección.
	\item \textbf{RF 4:} Una interfaz que indique el progreso de cada fase del juego.
	\item \textbf{RF 5:} Uso del giroscopio del dispositivo, para simular tener gravedad correctamente orientada.
	\item \textbf{RF 6:} Posibilidad de mover el dispositivo, alrededor del objeto representado por realidad aumentada.
	\item \textbf{RF 7:} Ofrecer información educativa implementada dentro del juego.
	\item \textbf{RF 8:} Ofrecer una Ayuda de usuario o tutorial.
	\item \textbf{RF 9:} Poder cambiar de idioma.
	\item \textbf{RF 10:} Poder hacer capturas de pantalla.
\end{itemize}
\subsection{Requisitos no Funcionales}
\begin{itemize}
	\item \textbf{RNF 1:} La aplicación debe de resultar fácil de usar y intuitiva para el usuario.
	\item \textbf{RNF 2:} Debe tener un rendimiento aceptable, para un uso cómodo.
	\item \textbf{RNF 3:} Debe de ser responsiva, pudiéndose adaptar la interfaz, a la pantalla de cualquier dispositivo.
\end{itemize}
\section{Especificación de requisitos}
\subsection{Descripción de los casos de uso}
%\tablaCasosDeUso{Caso de Uso 1: Menu}{}{}{}
\subsubsection{Caso de uso 1: Menú}
\begin{itemize}
	\item \textbf{Autor:} Saúl Martín Ibáñez
	\item \textbf{Descripción:} Se crea un menú en el que poder seleccionar distintos modos de juego que pueda haber en la aplicación. Así como para poder configurar algunas de normas o preferencias del usuario.
	\item \textbf{Precondiciones:} Compatibilidad del dispositivo con la aplicación.
	\item \textbf{Requisitos:} RF-1.1, RF-1.2, RF-1.3
	\item \textbf{Acciones:}
	\begin{enumerate}
		\item Abrir aplicación.
		\item Seleccionar modo de juego.
		\item Ajustar normas de juego a las preferencias que escoja el usuario o a las por defecto.
	\end{enumerate}
	\item \textbf{Post Condiciones:} Inicia el modo cámara ¿?
	\item \textbf{Excepciones:} No hay
	\item \textbf{Importancia:} media...
\end{itemize}
\subsubsection{Caso de uso 2: Botones Herramientas}
\begin{itemize}
	\item \textbf{Autor:} Saúl Martín Ibáñez
	\item \textbf{Descripción:} Se trata de los botones para poder seleccionar las diferentes opciones del modo jugable de la aplicación.
	\item \textbf{Precondiciones:}
	\item \textbf{Requisitos:} RF-2
	\item \textbf{Acciones:}
	\begin{enumerate}
		\item Seleccionar el botón de la acción deseada.
		\item Habilitar la herramienta seleccionada y deshabilitar la anterior. 
	\end{enumerate}
	\item \textbf{Post Condiciones:}
	\item \textbf{Excepciones:}
	\item \textbf{Importancia:} Alta
\end{itemize}
\subsubsection{Caso de uso 3: Atrás}
 \begin{itemize}
 	\item \textbf{Autor:} Saúl Martín Ibáñez
 	\item \textbf{Descripción:} Retroceder al menú, para poder volver a seleccionar otro modo.
 	\item \textbf{Precondiciones:} Estar en el segundo nivel del menú, o dentro del modo jugable.
 	\item \textbf{Requisitos:} RF-3
 	\item \textbf{Acciones:}
 	\begin{enumerate}
 		\item Pulsar el botón de Atrás.
 		\item Volver al menú principal.
 	\end{enumerate}
 	\item \textbf{Post Condiciones:} El usuario ha vuelto a la pantalla del menú.
 	\item \textbf{Excepciones:}	
 	\item \textbf{Importancia:} Media.
 \end{itemize}
\subsubsection{Caso de uso 4: Interfaz}
\begin{itemize}
	\item \textbf{Autor:} Saúl Martín Ibáñez
	\item \textbf{Descripción:} En la interfaz del juego se mostrara el progreso de cada fase, de esta manera el usuario, sabrá cuanto le falta para finalizar la fase en la que se encuentra.
	\item \textbf{Precondiciones:} Estar en la pantalla del modo jugable.
	\item \textbf{Requisitos:} RF-4
	\item \textbf{Acciones:}
	\begin{enumerate}
		\item Actualizar visualmente el progreso, cuando el usuario cumpla los requisitos.
	\end{enumerate}
	\item \textbf{Post Condiciones:} La interfaz se ha actualizado con éxito.
	\item \textbf{Excepciones:} La información de la interfaz no se actualiza.
	\item \textbf{Importancia:} media.
\end{itemize}
\subsubsection{Caso de uso 5: Giroscopio}
\begin{itemize}
	\item \textbf{Autor:} Saúl Martín Ibáñez
	\item \textbf{Descripción:} la aplicación puede usar el sensor giroscópico del terminal, para actualizar los vectores de gravedad de objetos de la aplicación.
	\item \textbf{Precondiciones:}-
	\item \textbf{Requisitos:} RF-5
	\item \textbf{Acciones:} No hay.
	\item \textbf{Post Condiciones:} Se actualizan los vectores de gravedad, según la inclinación del dispositivo.
	\item \textbf{Excepciones:} El dispositivo no posee del sensor o tiene inhabilitado los permisos.
	\item \textbf{Importancia:} Media.
\end{itemize}
\subsubsection{Caso de uso 6: }
 \begin{itemize}
 	\item \textbf{Autor:} Saúl Martín Ibáñez
 	\item \textbf{Descripción:} Al completar una fase ofrecerán datos educativos sobre esa fase. 
 	\item \textbf{Precondiciones:}...
 	\item \textbf{Requisitos:} RF-7
 	\item \textbf{Acciones:}
 	\item \textbf{Post Condiciones:} Se ha ofrecido información.
 	\item \textbf{Excepciones:} No se ha ofrecido ningún tipo de información de la fase.
 	\item \textbf{Importancia:} Media
 \end{itemize}
\subsubsection{Caso de uso 7: Ayuda}
\begin{itemize}
	\item \textbf{Autor:} Saúl Martín Ibáñez
	\item \textbf{Descripción:} Ofrecer una guía para los usuarios nobeles o que tengan dudas en las normas y funcionamiento de la aplicación.
	\item \textbf{Precondiciones:}
	\item \textbf{Requisitos:} RF-8
	\item \textbf{Acciones:}
	\begin{enumerate}
		\item Seleccionar botón de ayuda.
		\item En caso de estar en mitad del juego, pausar el progreso y poder continuar después.
		\item Desplegar una guía con las posibles dudas de utilización de la aplicación que puedan surgir.
	\end{enumerate}
	\item \textbf{Post Condiciones:}
	\item \textbf{Excepciones:} Tener la ayuda ya desplegada.
	\item \textbf{Importancia:} 
\end{itemize}
\subsubsection{Caso de uso 8: Cambiar Idioma}
\begin{itemize}
	\item \textbf{Autor:} Saúl Martín Ibáñez
	\item \textbf{Descripción:} Poder cambiar el idioma de los menús, y demás información textual de la aplicación, entre los idiomas disponibles.
	\item \textbf{Precondiciones:} No hay.
	\item \textbf{Requisitos:} RF-9
	\item \textbf{Acciones:}
	\begin{enumerate}
		\item Seleccionar botón de idioma.
		\item Seleccionar idioma deseado.
		\item Cambiar el idioma.
	\end{enumerate}
	\item \textbf{Post Condiciones:} El idioma de la aplicación ha cambiado satisfactoriamente.
	\item \textbf{Excepciones:} Ya se encuentra en ese idioma.
	\item \textbf{Importancia:} Baja.
\end{itemize}
\subsubsection{Caso de uso 9: Captura de Pantalla}
\begin{itemize}
	\item \textbf{Autor:} Saúl Martín Ibáñez
	\item \textbf{Descripción:} Poder realizar una foto del instante del juego con realidad aumentada.
	\item \textbf{Precondiciones:} Estar en el modo realidad aumentada.
	\item \textbf{Requisitos:} RF-10
	\item \textbf{Acciones:}
	\begin{enumerate}
		\item Presionar el botón de foto.
		\item Guardar la foto en la galería.
	\end{enumerate}
	\item \textbf{Post Condiciones:} La foto se ha guardado con éxito.
	\item \textbf{Excepciones:} La aplicación no tiene permisos para guardar un archivo.
	\item \textbf{Importancia:} Baja.
\end{itemize}

\subsection{Actores}
\begin{itemize}
	\item \textbf{Usuario:} usuario que hará uso aplicación.
\end{itemize}


