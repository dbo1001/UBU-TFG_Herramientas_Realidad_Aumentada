\apendice{Documentación técnica de programación}

\section{Introducción}
En este apartado se explicaran los pasos para la instalación y ejecución de las herramientas con las que se deberá trabajar.

\section{Estructura de directorios}
En este apartado se procederá a dar una breve explicación de la estructura que ha seguido los directorios del proyecto.\\

\dirtree{%
.1 /.
.2 Documentación/ Documentación: memoria y los anexos.
.3 img/ Imágenes de la documentación.
.3 text/ Archivos latex de la documentación.
.3 anexos.Pdf  Anexos.
.3 memoria.Pdf memoria.
.2 Ejemplo/ Ejemplo de realidad aumentada.
.3 Plant-test/ Ejemplo de realidad aumentada.
.4 Assest/ Assests del proyecto.
.5 Animation/ Animaciones usadas.
.5 AssestsOfStore/ Assests gratuitos descargados de la tiende de unity.
.5 Codigo/ Scripts usados.
.5 Material/ Materiales.
.5 Scenes/ Escenas de unity.
.4 Library/ Librerías usadas en el proyecto.
.3 APK/ Apk de la aplicación.
}



\section{Manual del programador}
\subsection{Unity}
Instalación.

Para Instalar el motor Unity es necesario descargar Unity Hub: \url{https://store.unity.com/es/download-nuo}

Unity Hub se trata de un launcher desde el que se pueden descargar varias versiones de Unity simultáneamente, pudiendo escoger la que mejor se acomode a las necesidades del usuario. También ofrece una serie de tutoriales para iniciarse en el desarrollo de Unity. Por último ofrece un listado de los proyectos del usuario, así como a que versión pertenecen, pudiendo escoger con que versión de Unity desean ejecutarlos.

En la pestaña de installs, pulsando el botón <<Add>>, saldrá una ventana donde poder escoger la versión deseada, al pasar el siguiente paso deberemos escoger los módulos complementarios para Unity, por el momento sería necesario el de Android, para poder pasar nuestros proyectos a una aplicación Android, importante desplegar las opciones del módulo y seleccionar ambas, pues para la compilación de una aplicación es estrictamente necesario el APK de Android.\imagen{unity01}{Paso 1: Seleccionar la versión de Unity }
\imagen{unity02}{Paso 2: Seleccionar}

Una vez terminada la instalación ya es posible comenzar a crear proyectos. 
Para poder descargar y usar Assets desde la tienda de Unity es necesario tener una cuenta de usuario.

Dado que en el ejemplo realizado en Unity se utiliza Vuforia como herramienta de realidad aumentada, es necesario añadir dicha herramienta al proyecto de Unity en el se trabajará. Para esto hay varios métodos posibles:
\begin{itemize}
\item Desde la pestaña de Edit-Project Settings, en el apartado Player-XR Settings, nos encontraremos con varias herramientas que se han instalado por defecto con Unity, marcamos el check de Vuforia y se incluirá automáticamente en nuestro proyecto.
\imagenPeque{VuforiaPlayer01}{Instalación Vuforia en Unity paso 1.}

\imagenPequeDos{VuforiaPlayer02}{Instalación Vuforia en Unity paso 2.}
\item Otra opción posible es, desde la pestaña de la tienda de Assests de Unity, podemos buscar Vuforia Core Samples. Aquí la podremos encontrar de forma gratuita para descargar, y el único proceso que tendremos que seguir es presionar el botón de descargar y seguidamente el botón de importar, de esta manera Vuforia quedará importado en nuestro proyecto. Si seguimos estos pasos también se habrán incorporado al proyecto algunos ejemplos ya construidos de las distintas funcionalidades de la herramienta.

\item Por ultimo, con el SDK de Vuforia para Unity, el cual le podemos descargar desde \href{https://developer.vuforia.com/downloads/sdk}{developer Vuforia} si tenemos una cuenta de usuario (Una cuenta gratuita es suficiente). Una vez que le hemos descargado, podemos importarlo en el proyecto ejecutándolo mientras tenemos el proyecto Unity abierto, o simplemente arrastrando el ejecutable dentro del proyecto abierto.	
\end{itemize}


\section{Compilación, instalación y ejecución del proyecto}

El proceso de instalación es sencillo. Una vez descargada la apk en el terminal, se procederá a su ejecución, el propio SO seguirá los pasos necesarios. La única intervención a resaltar, es posible que al tratar de instalar, nos pregunte si deseamos permitir la instalación de origen desconocido, esto se debe a que la apk no se encuentra oficialmente registrada el appstore. Una vez dados los permisos la instalación seguirá normalmente.

\section{Pruebas del sistema}
Las pruebas realizadas a la aplicación, se han realizado mediante pruebas de forma manual directamente en el dispositivo.

Una de las primeras pruebas ha realizar es el de la instalación en el dispositivo, para comprobar que el APK se ha realizado correctamente y es funcional. 

Para comprobar el re escalado de la interfaz de la aplicación, se ha probado en distintos dispositivos, varios smartphone, y una tablet. En cada pantalla de la interfaz, se comprueba que cada elemento es visible independientemente de el dispositivo.
En la interfaz del modo juego, en los dispositivos que tienen en la pantalla notch(cámara frontal) se aprecio que la interfaz no tenia en cuenta el notch, por lo que este tapaba una parte de la barra de progreso. Para corregirlo se ha bajado la barra de progreso.

En la detección de realidad aumentada, la gran mayoría de las pruebas se han podido hacer desde el entorno de Unity con una web cam. Pero para dar mayor fiabilidad a las pruebas, se han probado también en el dispositivo. Las pruebas consistían en comprobar que el mergecube era reconocido correctamente, y que cuando es detectado, el modelo 3D asociado se puede visualizar adecuadamente. Para la prueba de detección del mergecube se enfoca en cada cara del cubo, demostrando que en cualquier cara es reconocido, para complementar se han hecho pruebas tapando un porcentaje de la cara del cubo. (VÍDEO)

También se ha probado la detección con diferentes entornos de luminosidad, en el caso de los extremos de luminosidad(poca y mucha), la detección del mergecube se dificulta. Por ejemplo en los casos de demasiada iluminación, provoca que el contraste de iluminación impida reconocer el mergecube. (REFERENCIA A VÍDEO DEMOSTRACIÓN)

Para las pruebas de navegación entre el menú, se ha procedido ha probar que cada botón redirige a su pantalla correspondiente.

Para la comprobación del uso del sensor del giroscopio, seleccionamos la herramienta de la regadera, una vez seleccionada esta empieza a soltar partículas de agua, al girar el dispositivo en diferentes ángulos las partículas se moverán para dar la sensación de que siempre están cayendo hacia el suelo. (VIDEO)







Por ultimo destacar que Unity posee cualidades para realizar algunas pruebas unitarias de forma automática, pero no he tenido para aprender a utilizarlas de la manera correcta para esta aplicación. 