\apendice{Documentación de usuario}

\section{Introducción}
 En este apartado se explicarán, los requisitos que necesita el usuario para ejecutar la aplicación, los pasos para su instalación, y cómo usarla.
  
\section{Requisitos de usuarios}
 El ejemplo desarrollado se trata de una aplicación Android, por lo que el primer requisito necesario es disponer de un Dispositivo Android, con una versión Android 6.0 (Marshmallow) o superior.
  
 Ademas deberá cumplir los siguientes requisitos\cite{vuforia_devices}:
 \begin{itemize}
 	\item CPU Quad Core.
 	\item 1-2GB de memoria RAM.
 	\item Sensor de Giróscopio.
 	\item Se recomienda una cámara que permita grabar mínimo en 720p.  
 \end{itemize}
También es necesario disponer de un MergeCube. Para utilizar el Mergecube tenemos las opciones de comprar el original o construirse uno de papel. En ambos casos se han probado y son validados, la pega del de papel es su fragilidad y que dependiendo de la calidad de la impresión los resultados pueden variar.

\section{Instalación}

Dado que la aplicación no se ha publicado en Play Store, se deberá descargar el APK desde (\href{https://github.com/smi0010/TFG_Herramientas_Realidad_Aumentada/releases/tag/UBU_AR}{UBU AR}).
La instalación de la aplicación es sencilla, como la de cualquier otra APK. Una vez descargado el APK en el dispositivo, ejecutarle y seguir los pasos. 

Dado que la aplicación no se ha descargado desde una fuente oficial, como podría ser la Play Store, es posible que en el primer intento de instalación, el sistema solicite habilitar la opción de instalar aplicaciones de origen desconocido, que por defecto en los dispositivos viene desactivada.
En un principio los propios pasos de instalación permite habilitar la opción. En caso contrario deberás buscar en opciones del sistema. En caso de tener complicaciones para encontrar la ubicación de dicha opción, es recomendable realizar una sencilla búsqueda en Google, de ``Activar la instalación de aplicaciones de fuentes desconocidas en <<Modelo del dispositivo usado>>, con <<Versión de Android que posee>>``.


\section{Manual del usuario}

En este apartado se explicará los pasos básicos para aprender a usar la aplicación.

En la pantalla principal, al ejecutar la aplicación, encontraremos un menú con 4 cuatro opciones: 
\begin{itemize}
	\item Modos de Juego
	\item Opciones
	\item Ayuda
	\item Info
\end{itemize}


En la primera opción ``\textbf{Modos de Juego}'' el usuario podrá buscar entre los diferentes niveles disponibles la aplicación y seleccionar uno, ejecutando así el modo jugable. en la imagen (ref).

En ``\textbf{Opciones}'', se puede configurar diferentes aspectos de la aplicación, como el tipo de target que se usará en la realidad aumentada. En la imagen~(\ref{fig:Anexos/appOpciones}) podemos ver las opciones disponible.

\imagen{Anexos/appOpciones}{Imagen perteneciente a las opciones de la aplicación.}

En la opción ``\textbf{Ayuda}'', se ofrece una guía/tutorial en la que se explican los pasos básicos para el uso de la aplicación y poder resolver las dudas de los usuarios que se puedan encontrar.

La cuarta y ultima Opción se trata de una venta en la que se muestra un resumen respecto al TFG y la aplicación. Se detallan datos como el autor, lo tutores del TFG. 


\subsection{Modo MergeCube }

Al seleccionar uno de los modelos de planta teniendo seleccionado el modo con el MergeCube (ver figura~\ref{fig:Anexos/appOpciones}), la aplicación cambiará al modo de realidad aumentada activando la cámara del dispositivo. Ahora tendremos una interfaz que mostrará una barra de progreso en la parte superior, un contador del tiempo restante para acabar la fase en la parte superior derecha. En la parte inferior derecha hay un botón, que al pulsarse, despliega una barra con las herramientas disponibles para completar las fases. En la imagen \ref{fig:Anexos/appInterfaz} podemos ver unas muestras de la interfaz.

En este modo, una vez que la cámara ya está activa, al enfocar el Mergecube, será detectado y se cargará el modelo de realidad aumentada. En este punto es importante recordar, que para una correcta detección de un marcador, se precisa de estar en un ambiente con una iluminación adecuada para que el marcador no se vea afectado por sombras o focos de luz que reduzcan su visibilidad para el sensor de la cámara.

\imagen{Anexos/appInterfaz}{Imagen perteneciente a la aplicación, con la interfaz desplegada.}

\subsection{Modo CapturaTarget}

En el caso de tener seleccionado el \textbf{modo de capturar target}~(ver figura~\ref{fig:Anexos/appOpciones}), la interfaz contará con dos elementos adicionales: 
\begin{itemize}
	\item El primero se trata de una barra de colores (rojo, amarillo, verde) que indica la calidad óptima para que el objetivo enfocado sea un target. En la imagen~\ref{fig:Anexos/appCapturaTarget} se puede observar como al apuntar a una fondo blanco indica que tiene mala calidad, y al apuntar un fondo con una composición de elementos y colores buena la indica que es optimo para ser un target.
	\item El segundo elemento será el disparador situado en la parte inferior central. Una vez que tengamos un objetivo que cumpla los requisitos de calidad, podremos presionar el disparador, de manera que el target se guardará temporalmente (al cerrar la aplicación o volver al menú los target se reinician). 
\end{itemize}

Cuando capturamos un target, el disparador y la barra de colores desaparecerán y saltará un mensaje preguntándonos si deseamos continuar con dicho target. Si seleccionamos que ``no``, tendremos que volver a capturar un nuevo target. Sobre este target capturado aparecerá el modelo 3D correspondiente y ahora se podrá ``jugar`` del mismo modo que se hacia con el Mergecube. 

En el modo jugable el número de target simultáneos esta limitado a uno, si se desea capturar un nuevo target se deberá indicar presionando el botón de la parte inferior izquierda, donde nos preguntará si deseamos continuar con el target, al presionar que ``no``, volverá a salir las interfaz de captura para tomar un nuevo target. Al capturar un nuevo target sustituirá al anterior, pero podremos seguir con el mismo progreso del nivel que teníamos con el anterior target. 

\imagen{Anexos/appCapturaTarget}{Imagen pertenecientes a la captura de un target.}

\subsection{Barra de Herrameintas}
En la barra de herramientas contamos con dos elementos, una regadera y un paquete de abono. La primera es un icono de gotas de agua, al seleccionar nos saldrá en pantalla una regadera que automáticamente echará agua, el objetivo es mover el dispositivo hasta conseguir que el agua caiga en el Mergecube. Cuando el agua colisione con el modelo proyectado en el Mergecube, la barra de progreso subirá hasta completarse. 

El segundo elemento, se trata de un bote de fertilizante, al pulsarse saldrá en la pantalla un botón que al ser presionado saldrá delante como si se hubiese lanzado contra el Mergecube.
Para completar una fase es necesario haber completado ambos casos anteriores descritos.

También en la barra de herramientas aparecen dos opciones, la de Activar el Device Tracker con el botón ``Fijar posición``, y por último la opción de sacar una captura de pantalla.

\imagenPeque{Anexos/barraHerramienta}{Imagen perteneciente a la barra de herramientas disponibles.}

Actualmente los modos de juego de las plantas constan de cinco fases~(ver imagen~\ref{fig:Anexos/appFases}), cada una correspondiente con un estado de la planta (semilla, brote, planta pequeña, planta mediana, planta final). Para completar una fase, es necesario, completar la barra de progreso de agua, y haber echado fertilizante. Si el fertilizante se ha utilizado en una fase se indica con una luz verde en la barra de progreso.

\imagen{Anexos/appFases}{Imagen perteneciente a las fases de la planta de tomate.}

\subsection{Modo Múltiples Marcadores}
En el modo de captura target, también cuenta con un modo de múltiples marcadores. En este modo no se puede ``jugar``, pero es posible tener múltiples marcadores personalizados al mismo tiempo. Esto da la posibilidad de poder observar varios los modelos de la aplicación al mismo tiempo, como podemos ver en la imagen~\ref{fig:Anexos/appMultiTarget}. Por defecto la aplicación esta limitada a cinco marcadores simultáneos como máximo, para asegurar el rendimiento. En caso de realizar una sexta captura borrará la primera.

\imagenMasPeque{Anexos/appMultiTarget}{Imagen perteneciente al modo de múltiples marcadores simultáneos.}

En este modo podemos cambiar el tamaño de los modelos 3D, para ello primero debemos tocar el modelo que queramos modificar. Y para modificar unicamente hay que pulsar los botones de la parte interior izquierda.
En las opciones de la parte inferior derecha, podemos seleccionar que modelo 3D queremos que se asocie al marcador que vamos a capturar.


 

