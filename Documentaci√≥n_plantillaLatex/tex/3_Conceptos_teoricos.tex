\capitulo{3}{Conceptos teóricos}


\section{Realidad aumentada}
El concepto de realidad aumentada se aplica a las tecnologías que permiten añadir información gráfica sobrepuesta al mundo real en pantalla.
La realidad virtual se diferencia que esta crea un entorno virtual completo en el que el usuario se inmerge, mientras que la aumentada sobrepone información virtual sobre el mundo real.
 
origenes y evolucion...


\section{Herramientas AR}
He estado comparando diferentes herramientas de desarrollo para realidad aumentada. Algunas de ellas se tratan de herramientas ya desarrolladas y usadas con fines educativos o comerciales. Otras de las herramientas que veremos son los frameworks/api que muchas de las herramientas anteriores usan.
nuevas online, poder la web.. 

\subsection{CoSpaces}
Es una plataforma de apoyo educativo. Se puede acceder a ella tanto por web como desde la aplicación de móvil. Permite a los profesores crear salas para sus alumnos donde tendrían acceso a los diferentes ejemplos a usar para sus clases. También cuenta con una galería donde los usuarios pueden compartir sus ejemplos.

Para crear los proyectos se pueden hacer desde la web donde tiene un entorno de desarrollo 3D donde puedes añadir paredes, objetos etc , para tu escenario. Pero esto a limitado a un número de objetos que se pueden añadir al proyecto. Para añadir más hay que tener la versión premium. También puedes codificar eventos, acciones. mediante programación de bloques, también se puede mediante scrips pero esta opción pertenece a la versión premium. Para añadir modelos 3D al proyecto se puede subir desde un archivo local o buscando google poly(biblioteca publica de google de modelos3D)
Tienen compatibilidad con el mergecube para diseñar proyectos para el cubo, aunque este plugin se encuentra disponible para la versión premium.

...

\subsection{Metaverse}
Es otra aplicación enfocada bastante al entorno educativo. En este caso el entorno de desarrollo de los ejemplos es más sencillo. En la propia web, tendremos una estructura similar un modelo de diagrama de flujo. En cada paso se puede añadir un objeto 3d o una imagen que es la que estará flotando cuando estemos usando la realidad aumentada, luego se puede añadir botones y menus, para que te lleven a otra pantalla que tenga otro objeto asignado. Así por ejemplo he visto ejemplo en los que hace pequeños juegos de preguntas y dependiendo de las respuestas te llevan a diferentes pantallas.
También da la posibilidad de importar tus propios modelos y imágenes.
También incluye la posibilidad de reconocer expresiones faciales para poder usarlo como disparadores.
También puede usar el Arcore y Arkit para permitir que los objetos puedan asentarse en una posición y poder girar alrededor suyo.

\subsection{Zapworks}
Esta aplicación esta se podría considerar que esta más enfocada a lo comercial. Ofrece una prueba gratuita de 30 días, luego hay diferentes modelos de suscripción.
Por lo que he estado viendo ofrece más posibilidades que las anteriores herramientas. Pero también es mas compleja ya que usa su la aplicación para diseño y programación de ejemplos es mas completa y ofrece mas posibilidades.
Para trabajar con los proyectos tiene su propia aplicación de escritorio, desde donde se puedes crear tus objetos, mecánicas que quieras incorporar, programar los scripts (javascript) a usar.

Algunos ejemplos que veo son de poder reconocer por ejemplo un folleto de publicidad que sería el disparador y que salte un modelo 3d para hacer un anuncio interactivo, o lo mismo reconociendo una imagen del sistema solar y que salte un modelo 3D del sistema solar. 
Otros en los que reconoce la cara(facetracking) y es capaz de poner objetos en AR, por ejemplo, sombreros, gafas, cascos…
Tiene también posee Word tracking, que en resumen es la capacidad de reconocer el entorno para poder colocar objetos de una manera mas realista sobre el terreno, por ejemplo, que de la sensación de que realmente esta sobre la mesa aunque vayamos moviendo el ángulo de la cámara.

\subsection{kudan}
Es un tecnología para percepción artificial, la cual se puede utilizar para la AR/Vr , robótica y drones, y los coches.
Usa la tecnología slam (Simultaneous Localization and Mapping) para el reconocimiento del escenario, esta tecnología también la usa ArCore por ejemplo.
Tiene disponible SDK para desarrollar, en iOS, Android, y Unity.

\subsection{Vuforia}
Es un SDk para realizar aplicaciones de realidad aumentada. Se puede emplear Java, C++ y desarrollo con unity. Tiene diferentes licencias de pago, parece que también cuenta con una gratuita, pero con limitaciones.
Mapeamiento via OpenGL
Smart TerrainTM, capacidad de reconstruir un terreno en tiempo real, creando un mapa en 3D.

\subsection{Wikitude}


\subsection{ArUco}
Librería de código abierto, en c++, y requiere OpenCV para el reconocimiento de marcadores.
Por si mima no cuenta con la posibilidad de proyectar modelos tridimensionales, pero puede hacer uso de otras librerías para ello, como OpenGl.
Principalmente trata la detección de marcadores, para situar los elementos de AR.

\subsection{OpenCV}
Es una librería de visión por computación y machin learning, de código abierto. Permite identificar objetos, caras, hacer tracking..
En este caso no lo tengo muy claro la herramienta, pero he visto casos que únicamente utilizan OpenCv, pero también otros en que lo combinan con Arkit, ArUco u otros para mejorar los resultados.

\subsection{ArCore}
\subsection{ArKit}
\subsection{8thWall}
Entorno de desarrollo, que se caracteriza por que da la posibilidad de ejecutar el proyecto desde el navegador web dándole permiso para usar a la cámara del dispositivo, de forma que no haría falta instalarse ninguna aplicación.


\section{Tecnicas AR}
	\subsection{Detección de marcadores}
	\subsubsection{Mergecube}
	\subsection{Detección del entorno}
	\subsubsection{SLAM}
	Slam,Simultaneous Localization and Mapping, se trata de una tecnica de reconocimiento del escenario, puede reconocer donde hay una superficie plana, donde un pared
	
	
	
\section{Secciones}

Las secciones se incluyen con el comando section.

\subsection{Subsecciones}

Además de secciones tenemos subsecciones.

\subsubsection{Subsubsecciones}

Y subsecciones. 


\section{Referencias}

Las referencias se incluyen en el texto usando cite \cite{wiki:latex}. Para citar webs, artículos o libros \cite{koza92}.


\section{Imágenes}

Se pueden incluir imágenes con los comandos standard de \LaTeX, pero esta plantilla dispone de comandos propios como por ejemplo el siguiente:

\imagen{escudoInfor}{Autómata para una expresión vacía}



\section{Listas de items}

Existen tres posibilidades:

\begin{itemize}
	\item primer item.
	\item segundo item.
\end{itemize}

\begin{enumerate}
	\item primer item.
	\item segundo item.
\end{enumerate}

\begin{description}
	\item[Primer item] más información sobre el primer item.
	\item[Segundo item] más información sobre el segundo item.
\end{description}
	
\begin{itemize}
\item 
\end{itemize}

\section{Tablas}

Igualmente se pueden usar los comandos específicos de \LaTeX o bien usar alguno de los comandos de la plantilla.

\tablaSmall{Herramientas y tecnologías utilizadas en cada parte del proyecto}{l c c c c}{herramientasportipodeuso}
{ \multicolumn{1}{l}{Herramientas} & App AngularJS & API REST & BD & Memoria \\}{ 
HTML5 & X & & &\\
CSS3 & X & & &\\
BOOTSTRAP & X & & &\\
JavaScript & X & & &\\
AngularJS & X & & &\\
Bower & X & & &\\
PHP & & X & &\\
Karma + Jasmine & X & & &\\
Slim framework & & X & &\\
Idiorm & & X & &\\
Composer & & X & &\\
JSON & X & X & &\\
PhpStorm & X & X & &\\
MySQL & & & X &\\
PhpMyAdmin & & & X &\\
Git + BitBucket & X & X & X & X\\
Mik\TeX{} & & & & X\\
\TeX{}Maker & & & & X\\
Astah & & & & X\\
Balsamiq Mockups & X & & &\\
VersionOne & X & X & X & X\\
} 
