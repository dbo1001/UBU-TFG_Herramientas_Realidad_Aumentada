\capitulo{3}{Conceptos teóricos}


\section{Realidad aumentada y virtual}
El concepto de realidad aumentada (AR Augmented Reality) se aplica a las tecnologías que permiten añadir información gráfica sobrepuesta al mundo real en pantalla.
La realidad virtual(RV) se diferencia de la anterior, en que esta crea un entorno virtual completo en el que el usuario se sumerge.
 
origenes y evolucion...




\section{Herramientas AR}
Se han comparado diferentes herramientas de desarrollo para realidad aumentada. Algunas de ellas se tratan de herramientas ya desarrolladas y usadas con fines educativos o comerciales. Otras de las herramientas que se ven son los frameworks/api que muchas de las herramientas anteriores usan.
nuevas online, poder la web.. 

\subsection{ArCore}
\subsection{ArKit}
\subsection{CoSpaces} CoSpaces\footnote{\url{https://cospaces.io/edu/}} es una plataforma de apoyo educativo. Se puede acceder a ella tanto por web como desde la aplicación de móvil. Permite a los profesores crear salas para sus alumnos donde tendrían acceso a los diferentes ejemplos a usar para sus clases. También cuenta con una galería donde los usuarios pueden compartir sus ejemplos.
\imagen{cospaces2}{Plataforma de desarrollo de CoSpaces}
Para la creación de proyectos, se puede hacer desde la web mediante un entorno de desarrollo 3D donde puedes añadir paredes, objetos y mas tipos de modelos3D, para tu escenario. Con una licencia gratuita, permite guardar dos proyectos como máximo al mismo tiempo. En los proyectos también es posible codificar eventos, acciones mediante una programación de bloques o scrips, aun que hay que destacar, que la programación por scrips y algunas opciones de la programación de bloques unicamente están disponibles con una licencia premium. Para añadir un modelo3D, que Cospace no ofrezca por defecto, se pueden subir desde un archivo local o desde la búsqueda integrada de CoSpace en Google Poly(biblioteca publica de google de modelos3D).
CoSpace también cuenta con la opción de diseñar proyectos para el mergecube, aunque este plugin se encuentra disponible para la versión premium.

...

\subsection{Metaverse} Metaverse\footnote{\url{https://studio.gometa.io/discover/me}} es otra aplicación enfocada bastante al entorno educativo. En este caso el entorno de desarrollo de los ejemplos es más sencillo. En la propia web, tendremos una estructura similar un modelo de diagrama de flujo. En cada paso se puede añadir un objeto 3D o una imagen que es la que estará flotando cuando estemos usando la realidad aumentada, posteriormente se pueden añadir botones y menús, para que te lleven a otra pantalla que tenga otro objeto asignado. Así, por ejemplo, se pueden hacer ejemplos en los que hace pequeños juegos de preguntas y, dependiendo de las respuestas, te llevan a diferentes pantallas.
También da la posibilidad de importar tus propios modelos e imágenes.
También incluye la posibilidad de reconocer expresiones faciales para poder usarlo como disparadores.
También puede usar el Arcore y Arkit para permitir que los objetos puedan asentarse en una posición y poder girar alrededor suyo.

\subsection{Zapworks} ZapWorks\footnote{\url{https://zap.works/}}
Esta aplicación se podría considerar que está más enfocada a lo comercial. Ofrece una prueba gratuita de 30 días, luego hay diferentes modelos de suscripción.
Ofrece más posibilidades que las anteriores herramientas. Pero también es mas compleja ya que usa su propia aplicación para diseño y programación de ejemplos////REVISARR////.
Para trabajar con los proyectos tiene su propia aplicación de escritorio, desde donde se pueden crear objetos, mecánicas que quieras incorporar, programar los scripts (javascript) a usar.

Algunos ejemplos que muestra son por ejemplo el reconocimiento de un folleto de publicidad que actúa de disparador y salta un modelo 3D del anuncio interactivo. Otro ejemplo en el que reconociendo una imagen del Sistema Solar consigue que se ejecute un modelo 3D del Sistema Solar. 
Otros en los que reconoce la cara (facetracking) y es capaz de poner objetos en AR, por ejemplo, sombreros, gafas, cascos…
Posee Word tracking, que en resumen es la capacidad de reconocer el entorno para poder colocar objetos de una manera mas realista sobre el terreno, por ejemplo, que de la sensación de que realmente esta sobre la mesa aunque vayamos moviendo el ángulo de la cámara.

\subsection{Kudan} Kudan\footnote{\href{https://www.xlsoft.com/en/products/kudan/?utm_source=google&utm_medium=adwords&utm_campaign=cp01&gclid=Cj0KCQiAvJXxBRCeARIsAMSkAprma7nGzsXVWd9w5H4HSQzNoOkF2eJu8LzAtEXbaDALYPUdfBOOgpEaAoqoEALw_wcB}{kudan}} es un tecnología para percepción artificial, la cual se puede utilizar para la AR/VR , robótica y drones, y los coches.
Usa la tecnología SLAM (Simultaneous Localization and Mapping) para el reconocimiento del escenario, esta tecnología también la usa ArCore por ejemplo.
Tiene disponible SDK para desarrollar, en iOS, Android, y Unity.

\subsection{Vuforia}
Vuforia\footnote{https://developer.vuforia.com/} es un kit de desarrollo(SDK) para aplicaciones de AR, para las plataformas de Android, iOS, Windows y HoloLens. Fue creado por Qualcomm Connected Experience en 2010, y en 2015 PTC inc lo compro.\cite{simonetti2013vuforia}

Vuforia proporciona una API en C++, Java, Objective-C++ y los lenguajes de .NET mediante de Unity.

Vuforia soporta diferentes tipos marcadores, estos pueden ser 2D o 3D, también soporta múltiples marcadores simultáneamente , y reconocimiento del terreno sin marcadores.
\imagen{vuforia01}{Posibles marcadores de Vuforia}

Los marcadores se pueden crear desde la pagina web de developer vuforia, para ello deben escoger que tipo de marcador quieren crear y adjuntarle las imágenes que formaran dicho marcador. La pagina dará una calificación de 5 estrellas según la calidad de la imagen para ser un marcador. Una vez completado, da la opción de descargarlos el un archivo configurable de Unity que incluirá los marcadores en nuestro proyecto.



Cuenta con una licencia gratuita por defecto, con la limitación de poder tener un máximo de 100 vumarks y 1000 cloud targets.
Es posible ampliar a 3 tipos de tarifas, básica `por 42 mes, básica +cloud por 92, y la pro en la deben ponerse en contacto con ellos para su registro y uso.


\subsection{Wikitude}


\subsection{ArUco}


\subsection{OpenCV}
Es una librería de visión por computación y machine learning, de código abierto. Permite identificar objetos, caras, hacer tracking..

....., pero he visto casos que únicamente utilizan OpenCv, pero también otros en que lo combinan con Arkit, ArUco u otros para mejorar los resultados.


\subsection{8thWall}
Entorno de desarrollo, que se caracteriza por que da la posibilidad de ejecutar el proyecto desde el navegador web dándole permiso para usar a la cámara del dispositivo, de forma que no haría falta instalarse ninguna aplicación.




\section{Técnicas AR}
	\subsection{Detección de marcadores}
	La detección de marcadores, se trata de un técnica de realidad aumentada en el que los sistemas utilizan como referencia espacial para situar elemento que queremos superponer, una figura o imagen concreta, que ha sido especificada previamente. Los marcadores mas comunes y simples son los códigos QR, pues se tratan de imágenes diseñadas para que puedan ser reconocidas fácilmente por las cámaras.
	\imagen{ejemploMarcadorCasa}{Ejemplo un marcador sencillo}
	Con la mejora de las cámaras, también es posible utilizar como marcadores imágenes mas complejas, como por ejemplo un folleto publicitario o la foto de un planeta.
	También es posible que el marcador se trate de un objeto físico, como por ejemplo una lampara, o un coche.
	\imagen{ejemploMarcadorTigre}{Ejemplo donde una carta de un tigre es un marcador}
	Aunque gracias al avance de la calidad de las cámaras y el los software de reconocimiento, es posible utilizar marcadores mas complejos,esto también implicarán que será mas probable que falle, y por consecuencia se desencuandren las imágenes insertadas por AR, que los tiempos de detección sean mas largos.
	\colorbox{red}{MEJORAR y Completar si es posible}
	
	\subsubsection{Mergecube}
	El mergecube\footnote{\url{https://mergeedu.com/}} se trata de un cubo diseñado por Merge , el cual tiene grabado unos dibujos por sus 6 caras, los dibujos actúan de marcadores. Es posible mover el cubo al mismo tiempo que se usa, de esa manera se consigue una experiencia mas interactiva. Es por eso que el mergecube es ideal para aplicarlo en experiencias educativas.
	\imagen{mergecube}{Mergecube} 
	\subsection{Detección del entorno}
	La detección del entorno trata de poder reconocer el entorno que la cámara capta y ser capaz de sin la ayuda de un marcador, poder ubicar los modelos en el escenario.
	Para esto hay diferentes técnicas, en resumen estas técnicas lo que hacen es detectar los cambios de profundidad de la imagen, y con esos datos evaluar que objetos están mas o menos cerca, cuales son planos, si se trata de un objeto que no tiene profundidad(una pared)...
	\subsubsection{SLAM}
	Slam,Simultaneous Localization and Mapping, se trata de una técnica de reconocimiento del entorno que la cámara visualiza, puede reconocer donde hay una superficie plana, donde un pared
	
	
	
\section{Secciones}

Las secciones se incluyen con el comando section.

\subsection{Subsecciones}

Además de secciones tenemos subsecciones.

\subsubsection{Subsubsecciones}

Y subsecciones. 


\section{Referencias}

Las referencias se incluyen en el texto usando cite \cite{wiki:latex}. Para citar webs, artículos o libros \cite{koza92}.


\section{Imágenes}

Se pueden incluir imágenes con los comandos standard de \LaTeX, pero esta plantilla dispone de comandos propios como por ejemplo el siguiente:

\imagen{escudoInfor}{Autómata para una expresión vacía}



\section{Listas de items}

Existen tres posibilidades:

\begin{itemize}
	\item primer item.
	\item segundo item.
\end{itemize}

\begin{enumerate}
	\item primer item.
	\item segundo item.
\end{enumerate}

\begin{description}
	\item[Primer item] más información sobre el primer item.
	\item[Segundo item] más información sobre el segundo item.
\end{description}
	
\begin{itemize}
\item 
\end{itemize}

\section{Tablas}

Igualmente se pueden usar los comandos específicos de \LaTeX o bien usar alguno de los comandos de la plantilla.

\tablaSmall{Herramientas y tecnologías utilizadas en cada parte del proyecto}{l c c c c}{herramientasportipodeuso}
{ \multicolumn{1}{l}{Herramientas} & App AngularJS & API REST & BD & Memoria \\}{ 
HTML5 & X & & &\\
CSS3 & X & & &\\
BOOTSTRAP & X & & &\\
JavaScript & X & & &\\
AngularJS & X & & &\\
Bower & X & & &\\
PHP & & X & &\\
Karma + Jasmine & X & & &\\
Slim framework & & X & &\\
Idiorm & & X & &\\
Composer & & X & &\\
JSON & X & X & &\\
PhpStorm & X & X & &\\
MySQL & & & X &\\
PhpMyAdmin & & & X &\\
Git + BitBucket & X & X & X & X\\
Mik\TeX{} & & & & X\\
\TeX{}Maker & & & & X\\
Astah & & & & X\\
Balsamiq Mockups & X & & &\\
VersionOne & X & X & X & X\\
} 
